%!TEX root = SMA-book.tex

% Useful info on newcomand: https://tex.stackexchange.com/questions/117358/newcommand-argument-confusion

% --------------------------------------------------------------------------------------------------
% General 
% --------------------------------------------------------------------------------------------------

\newcommand{\cst}{C}
\newcommand{\eqDist}{\overset{d}{=}} % Equality in distribution 
\newcommand{\diag}{\textrm{diag}}
\newcommand{\law}{\textrm{law}}
\newcommand{\cv}{\mathsf{cv}}
\newcommand{\Cor}{\mathrm{Cor}}

% --------------------------------------------------------------------------------------------------
% Bayesian inverse problem: general setup 
% --------------------------------------------------------------------------------------------------

% General inverse problems.
\newcommand{\Par}{u}
\newcommand{\parSpace}{\mathcal{U}} % Parameter space
\newcommand{\dimPar}{D} % Parameter dimension 
\newcommand{\idxParDim}{d} % Primary symbol used to index each parameter dimension. 
\newcommand{\obs}{y} % Data observation (response) vector
\newcommand{\noise}{\epsilon} % Random variable representing noise, typically in additive noise model. 
\newcommand{\obsSpace}{\mathcal{Y}} % Output space
\newcommand{\dimObs}{P} % Dimension of data observation vector
\newcommand{\idxObsDim}{p} % Primary symbol used to index each data dimension. 
\newcommand{\fwd}{\mathcal{G}} % Forward model
\newcommand{\llik}{\mathcal{L}} % Log-likelihood 
\newcommand{\lik}{\mathsf{L}} % Likelihood 
\newcommand{\priorDens}{\pi_0} % Prior density 
\newcommand{\postDens}{\pi} % Unnormalized posterior density.
\newcommand{\lpost}{\llik^{\postDens}} % Log unnormalized posterior density.
\newcommand{\postDensNorm}{\overline{\pi}} % Normalized posterior density.
\newcommand{\normCst}{Z} % Normalizing constant for posterior density. 
\newcommand{\likPar}{\Sigma} % Likelihood parameter
\newcommand{\ParTrue}{\Par_{\star}}

% --------------------------------------------------------------------------------------------------
% Gaussian Process Emulators
% --------------------------------------------------------------------------------------------------

% Generic GP commands. 
\newcommand{\Ndesign}{N}
\newcommand{\GP}{\mathcal{GP}} % GP distribution.
\newcommand{\LNP}{\mathcal{LNP}} % Log-normal process distribution. 
\newcommand{\func}{f} % Generic function used for results that hold for any generic GP. 
\newcommand{\funcPrior}{\func_{0}}
\newcommand{\funcVal}[1][\Ndesign]{\varphi_{#1}}
\newcommand{\funcValExact}[1][\Ndesign]{\varphi_{#1}^{*}}
\newcommand{\gpMeanBase}{\mu} % The base notation used for GP mean (no sub/superscripts). 
\newcommand{\gpKerBase}{k}
\newcommand{\gpMeanPrior}{\gpMeanBase_0}
\newcommand{\gpKerPrior}{\gpKerBase_0}
\newcommand{\gpMean}[1][\Ndesign]{\gpMeanBase_{#1}}
\newcommand{\gpKer}[1][\Ndesign]{\gpKerBase_{#1}}
\newcommand{\nuggetSD}{\eta}
\newcommand{\kerMat}[1][\Ndesign]{K_{#1}}
\newcommand{\emLawPrior}[1]{{\Pi}^{#1}} % The prior law/distribution of an emulator.
\newcommand{\design}[1][\Ndesign]{\mathcal{D}_{#1}}
\newcommand{\designRand}[2][\Nbatch]{\mathcal{D}_{#1}^{#2}} % When the response is unobserved/random. 
\newcommand{\emLaw}[2][\design]{{\Pi}^{#2}_{#1}} % The conditional law/distribution of an emulator (may or may not be Gaussian). 
\newcommand{\target}{\mathsf{f}} % Target map for emulation
\newcommand{\targetEm}[1][\Ndesign]{f_{#1}} % Emulator for target quantity
\newcommand{\targetTraj}{f} % Trajectory of emulator for target map
\newcommand{\targetTrajProp}{\tilde{\targetTraj}} % Proposed value of target trajectory (e.g., in a Metropolis-Hastings scheme)
\newcommand{\emDist}[1][\Ndesign]{\nu_{#1}} % Distribution (law) of the emulator (a stochastic process)

% Exponentiated quadratic kernel.
\newcommand{\lengthscale}{\ell}
\newcommand{\margSD}{\gamma}

% Log-likelihood and forward model emulators. 
\newcommand{\fwdPrior}{\fwd_{0}}
\newcommand{\llikPrior}{\llik_{0}}
\newcommand{\funcEm}[1][\Ndesign]{\func_{#1}}
\newcommand{\fwdEm}[1][\Ndesign]{\fwd_{#1}}
\newcommand{\llikEm}[1][\Ndesign]{\llik_{#1}} 
\newcommand{\lpostEm}[1][\Ndesign]{\llik^{\postDens}_{#1}} 
\newcommand{\postEm}[1][\Ndesign]{\postDens_{#1}} 
\newcommand{\postNormEm}[1][\Ndesign]{\postDensNorm_{#1}} 
\newcommand{\llikEmFwd}[1][\Ndesign]{\llik^{\fwd}_{#1}}
\newcommand{\Em}[2][\Ndesign]{{#2}_{#1}}

% Mean and kernel functions for log-likelihood and forward model emulators. 
\newcommand{\emMeanPrior}[1]{\gpMeanPrior^{#1}} 
\newcommand{\emKerPrior}[1]{\gpKerPrior^{#1}}
\newcommand{\emMean}[2][\Ndesign]{\gpMeanBase^{#2}_{#1}} % First argument is number of design points, second is the function being emulated. 
\newcommand{\emKer}[2][\Ndesign]{\gpKerBase^{#2}_{#1}}
\newcommand{\accProbMHEm}[1][\Ndesign]{\accProbMH_{#1}}

% Posterior Approximations
\newcommand{\normCstEm}[1][\Ndesign]{\normCst_{#1}}
\newcommand{\postApproxEP}[1][\Ndesign]{\postDensNorm_{#1}^{\mathrm{EP}}}
\newcommand{\postApproxMean}[1][\Ndesign]{\postDens_{#1}^{\mathrm{mean}}}
\newcommand{\postApproxMarg}[1][\Ndesign]{\postDens_{#1}^{\mathrm{EL}}}
\newcommand{\postApproxNormMarg}[1][\Ndesign]{\postDensNorm_{#1}^{\mathrm{EL}}}
\newcommand{\postApproxNormMean}[1][\Ndesign]{\postDensNorm_{#1}^{\mathrm{mean}}}


\newcommand{\llikEmRdm}[2][\Ndesign]{{#2}_{#1}}
\newcommand{\fwdEmRdm}[2][\Ndesign]{{#2}_{#1}^{\fwd}}
\newcommand{\llikEmRdmDens}[1][\Ndesign]{\llikEmRdm[#1]{\postDens}} % Convenience function 
\newcommand{\fwdEmRdmDens}[1][\Ndesign]{\fwdEmRdm[#1]{\postDens}} % Convenience function 
\newcommand{\llikEmSampDensNorm}[1][\Ndesign]{\llikEmRdm[#1]{\postDensNorm}^{\mathrm{EP}}}



% Marginal approximations. 
\newcommand{\llikEmMarg}[2][\Ndesign]{{#2}_{#1}^{\text{marg}}}
\newcommand{\fwdEmMarg}[2][\Ndesign]{{#2}_{#1}^{\fwd \text{,marg}}}
\newcommand{\llikEmMargDens}[1][\Ndesign]{\llikEmMarg[#1]{\postDens}} 
\newcommand{\fwdEmMargDens}[1][\Ndesign]{\fwdEmMarg[#1]{\postDens}} 
\newcommand{\CovComb}[1][\Ndesign]{\Em[{#1}]{\likPar}^{\fwd}}

% Mean approximations. 
\newcommand{\llikEmMean}[2][\Ndesign]{{#2}_{#1}^{\text{mean}}}
\newcommand{\fwdEmMean}[2][\Ndesign]{{#2}_{#1}^{\fwd \text{,mean}}}
\newcommand{\llikEmMeanDens}[1][\Ndesign]{\llikEmMean[#1]{\postDens}} 
\newcommand{\fwdEmMeanDens}[1][\Ndesign]{\fwdEmMean[#1]{\postDens}} 

% Quantile approximations. 
\newcommand{\quantileProb}{\alpha}
\newcommand{\GaussianCDF}{\Phi}
\newcommand{\quantile}[1][\quantileProb]{q^{\quantileProb}}
\newcommand{\llikEmQ}[2][\Ndesign]{{#2}_{#1}^{\quantileProb}}
\newcommand{\fwdEmQ}[2][\Ndesign]{{#2}_{#1}^{\fwd, \quantileProb}}
\newcommand{\llikEmQDens}[1][\Ndesign]{\llikEmQ[#1]{\postDens}} 
\newcommand{\fwdEmQDens}[1][\Ndesign]{\fwdEmQ[#1]{\postDens}} 

% Joint marginal approximations. 
\newcommand{\llikEmJointMarg}[2][\Ndesign]{{#2}_{#1}^{\textrm{joint-marg}}}
\newcommand{\fwdEmJointMarg}[2][\Ndesign]{{#2}_{#1}^{\fwd \textrm{,joint-marg}}}

% Joint Monte Carlo within Metropolis-Hastings noisy MCMC approximations. 
\newcommand{\postSampleKernel}{\mathcal{K}}
\newcommand{\mcwmhJointLabel}{\textrm{mcwmh-joint}}
\newcommand{\mcwmhIndLabel}{\textrm{mcwmh-ind}}
\newcommand{\mhPseudoMargLabel}{\textrm{mh-pseudo-marg}}
\newcommand{\llikEmJointMCWMH}[2][\Ndesign]{{#2}_{#1}^{\mcwmhJointLabel}}
\newcommand{\fwdEmJointMCWMH}[2][\Ndesign]{{#2}_{#1}^{\fwd \textrm{,\mcwmhJointLabel}}}

% Design. 
\newcommand{\Nbatch}{B}
\newcommand{\Nrounds}{K_{\textrm{design}}}
\newcommand{\designIndex}{k}
\newcommand{\Naugment}{\Ndesign + \Nbatch}
\newcommand{\idxDesign}{n} % Primary symbol used to index each design point. 
\newcommand{\parMat}{U} % Generic matrix of input parameter points. 
\newcommand{\parMatCand}{\parMat_{\mathrm{cand}}} % Candidate points for discrete optimization 
\newcommand{\designIn}[1][\Ndesign]{U_{#1}} % Design inputs. 
\newcommand{\designOutLlik}[1][\Ndesign]{\ell_{#1}}
\newcommand{\designOutFwd}[1][\Ndesign]{g_{#1}}
\newcommand{\designOutLlikExact}[1][\Ndesign]{\ell^*_{#1}}
\newcommand{\designOutFwdExact}[1][\Ndesign]{g^*_{#1}}
\newcommand{\designBatchIn}{U_{\Nbatch}}
\newcommand{\designBatchLlik}{\ell_{\Nbatch}}
\newcommand{\designBatchFunc}{\varphi_{\Nbatch}}
\newcommand{\designBatchFwd}{g_{\Nbatch}}
\newcommand{\fwdEmCond}[2][\Ndesign]{\fwdEm[{#1}]^{#2}}
\newcommand{\acq}[1][\Ndesign]{\mathcal{A}_{#1}}
\newcommand{\qoiMap}{\phi}
\newcommand{\qoi}[1][\Ndesign]{\qoiMap_{#1}}

% Acquisition functions 
\newcommand{\labelAcq}[2][\Ndesign]{\acq[#1]^{\textrm{#2}}}
\newcommand{\maxvarLabel}{\textrm{max-var}}
\newcommand{\maxentLabel}{\textrm{max-ent}}
\newcommand{\intvarLabel}{\textrm{int-var}}
\newcommand{\intentLabel}{\textrm{int-ent}}
\newcommand{\maxexpvarLabel}{\textrm{max-exp-var}}
\newcommand{\maxexpentLabel}{\textrm{max-exp-ent}}
\newcommand{\intExpVarLabel}{\textrm{int-exp-var}}
\newcommand{\fwdintExpVarLabel}{\fwd \textrm{,}\intExpVarLabel}
\newcommand{\intExpEntLabel}{\textrm{int-exp-ent}}
\newcommand{\Ent}{H} % Entropy 
\newcommand{\weightDens}[1][\Ndesign]{\rho_{#1}}
\newcommand{\varInflation}{v}

% Basis function emulation. 
\newcommand{\basisVec}{\phi}
\newcommand{\basisWeight}{w}
\newcommand{\dimBasis}{R}
\newcommand{\idxBasis}{r}

% --------------------------------------------------------------------------------------------------
% Markov Chain Monte Carlo. 
% --------------------------------------------------------------------------------------------------

% Generic MCMC commands. 
\newcommand{\propDens}{q} % Proposal density
\newcommand{\propPar}{\tilde{\Par}} % Proposal density
\newcommand{\llikSamp}{\ell} 
\newcommand{\llikSampProp}{\tilde{\ell}} 
\newcommand{\llikSampDist}{\nu}
\newcommand{\CovProp}{\mathbf{C}} % Proposal covariance matrix
\newcommand{\sampleIndex}{k} % Generic index for samples. 
\newcommand{\NSample}{K} % Generic variable for algorithms that draw a set of samples (e.g., MCMC)
\newcommand{\indexSample}[2][\sampleIndex]{{#2}_{#1}}
\newcommand{\NMCMC}{\NSample} % Proposal covariance matrix
\newcommand{\accProbMH}{\alpha} % Metropolis-Hastings acceptance probability
\newcommand{\avgAccProbMH}{\overline{a}}
\newcommand{\accProbRatio}{r}
\newcommand{\likRatio}{\lambda}
\newcommand{\MarkovKernel}{P}
\newcommand{\mcmcIndex}{\sampleIndex}
% \newcommand{\indexMCMC}[2][\mcmcIndex]{{#2}_{#1}}
\newcommand{\indexMCMC}[2][\mcmcIndex]{\indexSample[{#1}]{#2}}
\newcommand{\AR}{\mathsf{A}} % Metropolis-Hastings accept-reject step

% --------------------------------------------------------------------------------------------------
% Dynamical Models. 
% --------------------------------------------------------------------------------------------------

\newcommand{\Time}{t}
\newcommand{\state}{x}
\newcommand{\odeRHS}{F}
\newcommand{\timeStart}{\Time_0}
\newcommand{\timeEnd}{\Time_1}
\newcommand{\stateIC}{\state_{\circ}}
\newcommand{\dimState}{S}
\newcommand{\stateIndex}{s}
\newcommand{\indexState}[2][\stateIndex]{{#2}^{({#1})}}
\newcommand{\solutionOp}{\mathcal{S}}
\newcommand{\obsOp}{\mathcal{H}}
\newcommand{\timeIndex}{\mcmcIndex}
\newcommand{\indexTime}[2][\timeIndex]{{#2}_{#1}}
\newcommand{\NTimeStep}{\NSample}
\newcommand{\forcing}{w}

% --------------------------------------------------------------------------------------------------
% Very Simple Ecosystem Model (VSEM)
% --------------------------------------------------------------------------------------------------

\newcommand{\stateV}{\state_{\textrm{v}}}
\newcommand{\stateR}{\state_{\textrm{r}}}
\newcommand{\stateS}{\state_{\textrm{s}}}
\newcommand{\dstateV}{\dot{\state}_{\textrm{v}}}
\newcommand{\dstateR}{\dot{\state}_{\textrm{r}}}
\newcommand{\dstateS}{\dot{\state}_{\textrm{s}}}
\newcommand{\NPP}{\textrm{NPP}}
\newcommand{\GPP}{\textrm{GPP}}
\newcommand{\alphaV}{\alpha_{\textrm{v}}}
\newcommand{\tauV}{\tau_{\textrm{v}}}
\newcommand{\tauR}{\tau_{\textrm{r}}}
\newcommand{\tauS}{\tau_{\textrm{s}}}
\newcommand{\LUE}{\ell}
\newcommand{\LAR}{r}
\newcommand{\KEXT}{\kappa}
\newcommand{\fracRespiration}{\gamma}
\newcommand{\LAI}{\textrm{LAI}}








