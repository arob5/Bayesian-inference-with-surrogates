% Latex macros for surrogate uncertainty propagation paper.
% Andrew Roberts

% --------------------------------------------------------------------------------------------------
% General 
% --------------------------------------------------------------------------------------------------

\newcommand{\cst}{C}
\newcommand{\proptoAdd}{\overset{\mathrm{add}}{\propto}} % Absorbs additive constants
\newcommand{\eqDist}{\overset{d}{=}} % Equality in distribution 
\newcommand{\diag}{\textrm{diag}}
\newcommand{\law}{\textrm{law}}
\newcommand{\cv}{\mathsf{cv}}
\newcommand{\Cor}{\mathrm{Cor}}

% --------------------------------------------------------------------------------------------------
% Bayesian inverse problem.
% --------------------------------------------------------------------------------------------------

% General inverse problems.
\newcommand{\Par}{u}
\newcommand{\parSpace}{\mathbb{U}} % Parameter space
\newcommand{\dimPar}{D} % Parameter dimension 
\newcommand{\idxParDim}{d} % Primary symbol used to index each parameter dimension. 
\newcommand{\obs}{y} % Data observation (response) vector
\newcommand{\noise}{\epsilon} % Random variable representing noise, typically in additive noise model. 
\newcommand{\obsSpace}{\mathbb{Y}} % Output space
\newcommand{\dimObs}{P} % Dimension of data observation vector
\newcommand{\idxObsDim}{p} % Primary symbol used to index each data dimension. 
\newcommand{\fwd}{\mathcal{G}} % Forward model
\newcommand{\llik}{\mathcal{L}} % Log-likelihood 
\newcommand{\lik}{\mathsf{L}} % Likelihood 
\newcommand{\priorDens}{\pi_0} % Prior density 
\newcommand{\postDens}{\pi} % Unnormalized posterior density.
\newcommand{\lpost}{\llik^{\postDens}} % Log unnormalized posterior density.
\newcommand{\postDensNorm}{\overline{\pi}} % Normalized posterior density.
\newcommand{\normCst}{Z} % Normalizing constant for posterior density. 
\newcommand{\likPar}{\Sigma} % Likelihood parameter
\newcommand{\ParTrue}{\Par_{\star}}

% --------------------------------------------------------------------------------------------------
% Surrogates
% --------------------------------------------------------------------------------------------------

% General notation for representing emulated quantity
\newcommand{\Em}[1]{{#1}_{\star}} .

% Emulator target
\newcommand{\target}{\mathsf{f}} % Target map for emulation
\newcommand{\targetEm}{\Em{f}} % Emulator for target quantity
\newcommand{\targetTraj}{f} % Trajectory of emulator for target map
\newcommand{\emDist}{\nu} % Predictive distribution (law) of the emulator (a stochastic process)
\newcommand{\emDistPrior}{\emDist_0} % Prior predictive distribution of emulator
\newcommand{\emE}{\E_{\emDist}} % Expectation wrt emulator distribution.
\newcommand{\emObs}{z} % The design set used to train the emulator.
\newcommand{\targetRange}{\mathbb{F}} % Codomain for the target function.

% Gaussian processes
\newcommand{\Ndesign}{N}
\newcommand{\GP}{\mathcal{GP}} % GP distribution.
\newcommand{\LNP}{\mathcal{LNP}} % Log-normal process distribution.
\newcommand{\gpMeanBase}{\mu} % The base notation used for GP mean (no sub/superscripts). 
\newcommand{\gpKerBase}{k}
\newcommand{\gpMeanPrior}{\gpMeanBase_0}
\newcommand{\gpKerPrior}{\gpKerBase_0}
\newcommand{\gpMean}[1][\Ndesign]{\gpMeanBase_{#1}}
\newcommand{\gpKer}[1][\Ndesign]{\gpKerBase_{#1}}

% Surrogate posterior approximations
\newcommand{\postEm}{\Em{\postDens}}
\newcommand{\postNormEm}{\Em{\postDensNorm}}
\newcommand{\normCstEm}{\Em{\normCst}}
\newcommand{\postApproxEP}{\Em{\postDensNorm}^{\mathrm{EP}}}
\newcommand{\postApproxMean}{\Em{\postDens}^{\mathrm{mean}}}
\newcommand{\postApproxEUP}{\Em{\postDens}^{\mathrm{EUP}}}
\newcommand{\postApproxEUPNorm}{\Em{\postDensNorm}^{\mathrm{EUP}}}
\newcommand{\postApproxNormMean}{\Em{\postDensNorm}^{\mathrm{mean}}}

% Likelihood approximations.
\newcommand{\likApproxEP}{\Em{\lik}^{\mathrm{EP}}}
\newcommand{\likApproxEUP}{\Em{\lik}^{\mathrm{EUP}}}

% --------------------------------------------------------------------------------------------------
% Proofs and Theoretical Sections
% --------------------------------------------------------------------------------------------------

% Variational formulation.
\newcommand{\qMeas}{Q} % Generic prob measure on parSpace (used in variational formulation)
\newcommand{\qDens}{q} % Density of qDens.
\newcommand{\qSpace}{\mathcal{Q}} % Space of densities or measures.
\newcommand{\loss}{\mathcal{L}}
\newcommand{\qDensOpt}{\qDens_{\mathrm{opt}}} % Solution of optimization problem
\newcommand{\qMeasOpt}{\qMeas_{\mathrm{opt}}}

% Analyzing EUP vs. EP
\newcommand{\jgap}{\Delta_\normCst} % The "Jensen gap" for the normalizing constant.

% Fully Bayesian/Kennedy and O'Hagan formulation.
\newcommand{\jointKOH}{\zeta} % Joint distribution over both surrogate and calibration parameters (as in KOH setup)
\newcommand{\postKOH}{\jointKOH^{\obs, \emObs}} % Posterior of the joint distribution
\newcommand{\condMargKOH}{\jointKOH^{\obs,\targetTraj}_{\Par}} % Marginal conditional of \Par, given \targetTraj and \obs.
\newcommand{\qCond}{T_{\qMeas}} % In the KOH context we have \qMeas(d\Par,d\targetTraj) = \emDist(d\targetTraj)\qCond(\targetTraj, d\Par)
\newcommand{\qSpaceCut}{\qSpace_{\mathrm{cut}}}

% Measure-theoretic details for proofs.
\newcommand{\emJoint}{\eta} % Joint measure over (u,f)
\newcommand{\BorelSig}{\mathcal{B}}
\newcommand{\emSpace}{\mathcal{F}}
\newcommand{\emSig}{\mathcal{A}} % Sigma algebra in function space for emulator
\newcommand{\refMeas}{\lambda} % Reference measure on parameter space

% --------------------------------------------------------------------------------------------------
% Markov Chain Monte Carlo. 
% --------------------------------------------------------------------------------------------------

% General
\newcommand{\sampleIndex}{k} % Generic index for samples. 
\newcommand{\NSample}{K} % Generic variable for algorithms that draw a set of samples (e.g., MCMC)
\newcommand{\indexSample}[2][\sampleIndex]{{#2}_{#1}}
\newcommand{\indexMCMC}[2][\mcmcIndex]{\indexSample[{#1}]{#2}}
\newcommand{\NMCMC}{\NSample} % Number MCMC samples
\newcommand{\accProbMH}{\alpha} % Metropolis-Hastings acceptance probability

% Proposal distribution.
\newcommand{\propDens}{q} % Proposal density
\newcommand{\propDist}{Q} % Proposal distribution (measure)
\newcommand{\propPar}{\tilde{\Par}} % Proposed parameter value
\newcommand{\targetTrajProp}{\tilde{\targetTraj}} % Proposed value of target trajectory

% Preconditioned Crank-Nicholson
\newcommand{\pcnCor}{\rho}

% --------------------------------------------------------------------------------------------------
% Dynamical Models. 
% --------------------------------------------------------------------------------------------------

\newcommand{\Time}{t}
\newcommand{\state}{x}
\newcommand{\odeRHS}{F}
\newcommand{\timeStart}{\Time_0}
\newcommand{\timeEnd}{\Time_1}
\newcommand{\stateIC}{\state_{\circ}}
\newcommand{\dimState}{S}
\newcommand{\stateIndex}{s}
\newcommand{\indexState}[2][\stateIndex]{{#2}^{({#1})}}
\newcommand{\solutionOp}{\mathcal{S}}
\newcommand{\obsOp}{\mathcal{H}}
\newcommand{\timeIndex}{\mcmcIndex}
\newcommand{\indexTime}[2][\timeIndex]{{#2}_{#1}}
\newcommand{\NTimeStep}{\NSample}
\newcommand{\forcing}{w}

% --------------------------------------------------------------------------------------------------
% Numerical example: linear Gaussian model
% --------------------------------------------------------------------------------------------------

\newcommand{\fwdLin}{G} % Linear forward model
\newcommand{\priorMean}{m_0} 
\newcommand{\priorCov}{C_0}
\newcommand{\postMean}{m}
\newcommand{\postCov}{C}
\newcommand{\noiseCov}{\Sigma}
\newcommand{\emBias}{r}
\newcommand{\emCov}{Q}
\newcommand{\postMeanEP}{\postMean^{\mathrm{EP}}}
\newcommand{\postMeanEUP}{\postMean^{\mathrm{EUP}}}
\newcommand{\postCovEP}{\postCov^{\mathrm{EP}}}
\newcommand{\postCovEUP}{\postCov^{\mathrm{EUP}}}
\newcommand{\epGain}{H} % = \postCov \fwdLin^\top \noiseCov^{-1}
\newcommand{\eupNoiseCov}{\tilde{\noiseCov}} % = \noiseCov + \emCov
\newcommand{\eupObs}{\tilde{\obs}} % = \obs - \emBias

% --------------------------------------------------------------------------------------------------
% Numerical example: Very Simple Ecosystem Model (VSEM)
% --------------------------------------------------------------------------------------------------

\newcommand{\stateV}{\state_{\textrm{v}}}
\newcommand{\stateR}{\state_{\textrm{r}}}
\newcommand{\stateS}{\state_{\textrm{s}}}
\newcommand{\dstateV}{\dot{\state}_{\textrm{v}}}
\newcommand{\dstateR}{\dot{\state}_{\textrm{r}}}
\newcommand{\dstateS}{\dot{\state}_{\textrm{s}}}
\newcommand{\NPP}{\textrm{NPP}}
\newcommand{\GPP}{\textrm{GPP}}
\newcommand{\alphaV}{\alpha_{\textrm{v}}}
\newcommand{\tauV}{\tau_{\textrm{v}}}
\newcommand{\tauR}{\tau_{\textrm{r}}}
\newcommand{\tauS}{\tau_{\textrm{s}}}
\newcommand{\LUE}{\ell}
\newcommand{\LAR}{r}
\newcommand{\KEXT}{\kappa}
\newcommand{\fracRespiration}{\gamma}
\newcommand{\LAI}{\textrm{LAI}}








