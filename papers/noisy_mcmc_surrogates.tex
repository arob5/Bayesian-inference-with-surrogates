\documentclass[12pt]{article}
\RequirePackage[l2tabu, orthodox]{nag}
\usepackage[main=english]{babel}
\usepackage[rm={lining,tabular},sf={lining,tabular},tt={lining,tabular,monowidth}]{cfr-lm}
\usepackage[T1]{fontenc}
\usepackage[utf8]{inputenc}
\usepackage[pdftex]{graphicx}
\usepackage{amsthm,amssymb,latexsym,gensymb,mathtools,mathrsfs}
\usepackage{epstopdf,enumitem,microtype,dcolumn,booktabs,hyperref,url,fancyhdr}
\usepackage{algorithm}
\usepackage{algpseudocode} % Note that this also loads algorithmicx
\usepackage{cleveref}
\usepackage{natbib}
\usepackage{bbm}
\usepackage{caption, subcaption} % Captions and sub-figures. 
\usepackage{fancyvrb} % For writing using verbatim font inline. 
% \usepackage[demo]{graphicx}

% For tables
\usepackage{array}
\usepackage{booktabs}

% Bibliography
\bibliographystyle{plainnat}

% Plotting
\usepackage{pgfplots}
\usepackage{xinttools} % for the \xintFor***
\usepgfplotslibrary{fillbetween}
\pgfplotsset{compat=1.8}
\usepackage{tikz}

% Tables. 
\usepackage{multirow}

% Local custom commands. 
%!TEX root = SMA-book.tex

% Useful info on newcomand: https://tex.stackexchange.com/questions/117358/newcommand-argument-confusion


% --------------------------------------------------------------------------------------------------
% General commands.  
% --------------------------------------------------------------------------------------------------

\newcommand{\todo}{\textbf{TODO}}


% --------------------------------------------------------------------------------------------------
% General Math.  
% --------------------------------------------------------------------------------------------------

% Common math commands. 
\newcommand*{\abs}[1]{\left\lvert#1\right\rvert}
\newcommand{\R}{\mathbb{R}}
\newcommand*{\suchthat}{\,\mathrel{\big|}\,}
\newcommand{\Exp}[1]{\exp\mathopen{}\left\{#1\right\}\mathclose{}}
\newcommand{\Log}[1]{\log\mathopen{}\left\{#1\right\}\mathclose{}}
\newcommand{\BigO}{\mathcal{O}}
\newcommand{\Def}{\coloneqq} % coloneqq comes from the mathtools package. 

\DeclareMathOperator*{\argmax}{argmax}
\DeclareMathOperator*{\argmin}{argmin}
\DeclarePairedDelimiterX\innerp[2]{(}{)}{#1\delimsize\vert\mathopen{}#2}
\DeclarePairedDelimiter{\ceil}{\lceil}{\rceil}

% Linear algebra.
\newcommand*{\norm}[1]{\left\lVert#1\right\rVert}
\newcommand{\bx}{\mathbf{x}}
\newcommand{\bw}{\mathbf{w}}
\newcommand{\oneVec}[1][]{\boldsymbol{1}_{#1}}
\newcommand{\idMat}{\mathbf{I}}
\newcommand{\bS}{\mathbf{S}}
\newcommand{\bK}{\mathbf{K}}
\newcommand{\ba}{\mathbf{a}}
\newcommand{\bb}{\mathbf{b}}

% Probability. 
\newcommand{\E}{\mathbb{E}}
\newcommand{\Var}{\mathrm{Var}}
\newcommand{\Cov}{\mathrm{Cov}}
\newcommand{\Prob}{\mathbb{P}}
\newcommand{\Gaussian}{\mathcal{N}}
\newcommand{\LN}{\text{LN}} % Log-normal distribution 
\newcommand{\given}{\mid} 


%!TEX root = SMA-book.tex

% Useful info on newcomand: https://tex.stackexchange.com/questions/117358/newcommand-argument-confusion

% --------------------------------------------------------------------------------------------------
% General 
% --------------------------------------------------------------------------------------------------

\newcommand{\cst}{C}
\newcommand{\eqDist}{\overset{d}{=}} % Equality in distribution 
\newcommand{\diag}{\textrm{diag}}
\newcommand{\law}{\textrm{law}}
\newcommand{\cv}{\mathsf{cv}}
\newcommand{\Cor}{\mathrm{Cor}}
\newcommand{\BorelSig}{\mathcal{B}}
\newcommand{\emSpace}{\mathcal{F}}
\newcommand{\emSig}{\mathcal{A}} % Sigma algebra in function space for emulator
\newcommand{\refMeas}{\lambda} % Reference measure on parameter space

% --------------------------------------------------------------------------------------------------
% Bayesian inverse problem: general setup 
% --------------------------------------------------------------------------------------------------

% General inverse problems.
\newcommand{\Par}{u}
\newcommand{\parSpace}{\mathsf{U}} % Parameter space
\newcommand{\dimPar}{D} % Parameter dimension 
\newcommand{\idxParDim}{d} % Primary symbol used to index each parameter dimension. 
\newcommand{\obs}{y} % Data observation (response) vector
\newcommand{\noise}{\epsilon} % Random variable representing noise, typically in additive noise model. 
\newcommand{\obsSpace}{\mathsf{Y}} % Output space
\newcommand{\dimObs}{P} % Dimension of data observation vector
\newcommand{\idxObsDim}{p} % Primary symbol used to index each data dimension. 
\newcommand{\fwd}{\mathcal{G}} % Forward model
\newcommand{\llik}{\mathcal{L}} % Log-likelihood 
\newcommand{\lik}{\mathsf{L}} % Likelihood 
\newcommand{\priorDens}{\pi_0} % Prior density 
\newcommand{\postDens}{\pi} % Unnormalized posterior density.
\newcommand{\lpost}{\llik^{\postDens}} % Log unnormalized posterior density.
\newcommand{\postDensNorm}{\overline{\pi}} % Normalized posterior density.
\newcommand{\normCst}{Z} % Normalizing constant for posterior density. 
\newcommand{\likPar}{\Sigma} % Likelihood parameter
\newcommand{\ParTrue}{\Par_{\star}}

% Variational formulation.
\newcommand{\qMeas}{Q} % Generic prob measure on parSpace (used in variational formulation)
\newcommand{\qDens}{q} % Density of qDens.
\newcommand{\qSpace}{\mathcal{Q}} % Space of densities or measures.
\newcommand{\loss}{\mathcal{L}}

% --------------------------------------------------------------------------------------------------
% Gaussian Process Emulators
% --------------------------------------------------------------------------------------------------

% Generic GP commands. 
\newcommand{\Ndesign}{N}
\newcommand{\GP}{\mathcal{GP}} % GP distribution.
\newcommand{\LNP}{\mathcal{LNP}} % Log-normal process distribution. 
\newcommand{\func}{f} % Generic function used for results that hold for any generic GP. 
\newcommand{\funcPrior}{\func_{0}}
\newcommand{\funcVal}[1][\Ndesign]{\varphi_{#1}}
\newcommand{\funcValExact}[1][\Ndesign]{\varphi_{#1}^{*}}
\newcommand{\gpMeanBase}{\mu} % The base notation used for GP mean (no sub/superscripts). 
\newcommand{\gpKerBase}{k}
\newcommand{\gpMeanPrior}{\gpMeanBase_0}
\newcommand{\gpKerPrior}{\gpKerBase_0}
\newcommand{\gpMean}[1][\Ndesign]{\gpMeanBase_{#1}}
\newcommand{\gpKer}[1][\Ndesign]{\gpKerBase_{#1}}
\newcommand{\nuggetSD}{\eta}
\newcommand{\kerMat}[1][\Ndesign]{K_{#1}}
\newcommand{\emLawPrior}[1]{{\Pi}^{#1}} % The prior law/distribution of an emulator.
\newcommand{\design}[1][\Ndesign]{\mathcal{D}_{#1}}
\newcommand{\designRand}[2][\Nbatch]{\mathcal{D}_{#1}^{#2}} % When the response is unobserved/random. 
\newcommand{\emLaw}[2][\design]{{\Pi}^{#2}_{#1}} % The conditional law/distribution of an emulator (may or may not be Gaussian). 
\newcommand{\target}{\mathsf{f}} % Target map for emulation
\newcommand{\targetEm}[1][\Ndesign]{f_{#1}} % Emulator for target quantity
\newcommand{\targetTraj}{f} % Trajectory of emulator for target map
\newcommand{\targetTrajProp}{\tilde{\targetTraj}} % Proposed value of target trajectory (e.g., in a Metropolis-Hastings scheme)
\newcommand{\emDist}[1][\Ndesign]{\nu_{#1}} % Distribution (law) of the emulator (a stochastic process)
\newcommand{\emE}{\E_{\emDist[]}} % Expectation wrt emulator distribution.
\newcommand{\emObs}{z} % The design set used to train the emulator.
\newcommand{\targetRange}{\mathsf{F}}
\newcommand{\emJoint}{\eta} % Joint measure over (u,f)
\newcommand{\jointKOH}{\zeta} % Joint distribution over both surrogate and calibration parameters (as in KOH setup)
\newcommand{\jgap}{\Delta_\normCst} % The "Jensen gap" for the normalizing constant.

% Exponentiated quadratic kernel.
\newcommand{\lengthscale}{\ell}
\newcommand{\margSD}{\gamma}

% Log-likelihood and forward model emulators. 
\newcommand{\fwdPrior}{\fwd_{0}}
\newcommand{\llikPrior}{\llik_{0}}
\newcommand{\funcEm}[1][\Ndesign]{\func_{#1}}
\newcommand{\fwdEm}[1][\Ndesign]{\fwd_{#1}}
\newcommand{\llikEm}[1][\Ndesign]{\llik_{#1}} 
\newcommand{\lpostEm}[1][\Ndesign]{\llik^{\postDens}_{#1}} 
\newcommand{\postEm}[1][\Ndesign]{\postDens_{#1}} 
\newcommand{\postNormEm}[1][\Ndesign]{\postDensNorm_{#1}} 
\newcommand{\llikEmFwd}[1][\Ndesign]{\llik^{\fwd}_{#1}}
\newcommand{\Em}[2][\Ndesign]{{#2}_{#1}}

% Mean and kernel functions for log-likelihood and forward model emulators. 
\newcommand{\emMeanPrior}[1]{\gpMeanPrior^{#1}} 
\newcommand{\emKerPrior}[1]{\gpKerPrior^{#1}}
\newcommand{\emMean}[2][\Ndesign]{\gpMeanBase^{#2}_{#1}} % First argument is number of design points, second is the function being emulated. 
\newcommand{\emKer}[2][\Ndesign]{\gpKerBase^{#2}_{#1}}
\newcommand{\accProbMHEm}[1][\Ndesign]{\accProbMH_{#1}}

% Posterior Approximations
\newcommand{\normCstEm}[1][\Ndesign]{\normCst_{#1}}
\newcommand{\postApproxEP}[1][\Ndesign]{\postDensNorm_{#1}^{\mathrm{EP}}}
\newcommand{\postApproxMean}[1][\Ndesign]{\postDens_{#1}^{\mathrm{mean}}}
\newcommand{\postApproxMarg}[1][\Ndesign]{\postDens_{#1}^{\mathrm{EUP}}}
\newcommand{\postApproxNormMarg}[1][\Ndesign]{\postDensNorm_{#1}^{\mathrm{EUP}}}
\newcommand{\postApproxNormMean}[1][\Ndesign]{\postDensNorm_{#1}^{\mathrm{mean}}}


\newcommand{\llikEmRdm}[2][\Ndesign]{{#2}_{#1}}
\newcommand{\fwdEmRdm}[2][\Ndesign]{{#2}_{#1}^{\fwd}}
\newcommand{\llikEmRdmDens}[1][\Ndesign]{\llikEmRdm[#1]{\postDens}} % Convenience function 
\newcommand{\fwdEmRdmDens}[1][\Ndesign]{\fwdEmRdm[#1]{\postDens}} % Convenience function 
\newcommand{\llikEmSampDensNorm}[1][\Ndesign]{\llikEmRdm[#1]{\postDensNorm}^{\mathrm{EP}}}



% Marginal approximations. 
\newcommand{\llikEmMarg}[2][\Ndesign]{{#2}_{#1}^{\text{marg}}}
\newcommand{\fwdEmMarg}[2][\Ndesign]{{#2}_{#1}^{\fwd \text{,marg}}}
\newcommand{\llikEmMargDens}[1][\Ndesign]{\llikEmMarg[#1]{\postDens}} 
\newcommand{\fwdEmMargDens}[1][\Ndesign]{\fwdEmMarg[#1]{\postDens}} 
\newcommand{\CovComb}[1][\Ndesign]{\Em[{#1}]{\likPar}^{\fwd}}

% Mean approximations. 
\newcommand{\llikEmMean}[2][\Ndesign]{{#2}_{#1}^{\text{mean}}}
\newcommand{\fwdEmMean}[2][\Ndesign]{{#2}_{#1}^{\fwd \text{,mean}}}
\newcommand{\llikEmMeanDens}[1][\Ndesign]{\llikEmMean[#1]{\postDens}} 
\newcommand{\fwdEmMeanDens}[1][\Ndesign]{\fwdEmMean[#1]{\postDens}} 

% Quantile approximations. 
\newcommand{\quantileProb}{\alpha}
\newcommand{\GaussianCDF}{\Phi}
\newcommand{\quantile}[1][\quantileProb]{q^{\quantileProb}}
\newcommand{\llikEmQ}[2][\Ndesign]{{#2}_{#1}^{\quantileProb}}
\newcommand{\fwdEmQ}[2][\Ndesign]{{#2}_{#1}^{\fwd, \quantileProb}}
\newcommand{\llikEmQDens}[1][\Ndesign]{\llikEmQ[#1]{\postDens}} 
\newcommand{\fwdEmQDens}[1][\Ndesign]{\fwdEmQ[#1]{\postDens}} 

% Joint marginal approximations. 
\newcommand{\llikEmJointMarg}[2][\Ndesign]{{#2}_{#1}^{\textrm{joint-marg}}}
\newcommand{\fwdEmJointMarg}[2][\Ndesign]{{#2}_{#1}^{\fwd \textrm{,joint-marg}}}

% Joint Monte Carlo within Metropolis-Hastings noisy MCMC approximations. 
\newcommand{\postSampleKernel}{\mathcal{K}}
\newcommand{\mcwmhJointLabel}{\textrm{mcwmh-joint}}
\newcommand{\mcwmhIndLabel}{\textrm{mcwmh-ind}}
\newcommand{\mhPseudoMargLabel}{\textrm{mh-pseudo-marg}}
\newcommand{\llikEmJointMCWMH}[2][\Ndesign]{{#2}_{#1}^{\mcwmhJointLabel}}
\newcommand{\fwdEmJointMCWMH}[2][\Ndesign]{{#2}_{#1}^{\fwd \textrm{,\mcwmhJointLabel}}}

% Design. 
\newcommand{\Nbatch}{B}
\newcommand{\Nrounds}{K_{\textrm{design}}}
\newcommand{\designIndex}{k}
\newcommand{\Naugment}{\Ndesign + \Nbatch}
\newcommand{\idxDesign}{n} % Primary symbol used to index each design point. 
\newcommand{\parMat}{U} % Generic matrix of input parameter points. 
\newcommand{\parMatCand}{\parMat_{\mathrm{cand}}} % Candidate points for discrete optimization 
\newcommand{\designIn}[1][\Ndesign]{U_{#1}} % Design inputs. 
\newcommand{\designOutLlik}[1][\Ndesign]{\ell_{#1}}
\newcommand{\designOutFwd}[1][\Ndesign]{g_{#1}}
\newcommand{\designOutLlikExact}[1][\Ndesign]{\ell^*_{#1}}
\newcommand{\designOutFwdExact}[1][\Ndesign]{g^*_{#1}}
\newcommand{\designBatchIn}{U_{\Nbatch}}
\newcommand{\designBatchLlik}{\ell_{\Nbatch}}
\newcommand{\designBatchFunc}{\varphi_{\Nbatch}}
\newcommand{\designBatchFwd}{g_{\Nbatch}}
\newcommand{\fwdEmCond}[2][\Ndesign]{\fwdEm[{#1}]^{#2}}
\newcommand{\acq}[1][\Ndesign]{\mathcal{A}_{#1}}
\newcommand{\qoiMap}{\phi}
\newcommand{\qoi}[1][\Ndesign]{\qoiMap_{#1}}

% Acquisition functions 
\newcommand{\labelAcq}[2][\Ndesign]{\acq[#1]^{\textrm{#2}}}
\newcommand{\maxvarLabel}{\textrm{max-var}}
\newcommand{\maxentLabel}{\textrm{max-ent}}
\newcommand{\intvarLabel}{\textrm{int-var}}
\newcommand{\intentLabel}{\textrm{int-ent}}
\newcommand{\maxexpvarLabel}{\textrm{max-exp-var}}
\newcommand{\maxexpentLabel}{\textrm{max-exp-ent}}
\newcommand{\intExpVarLabel}{\textrm{int-exp-var}}
\newcommand{\fwdintExpVarLabel}{\fwd \textrm{,}\intExpVarLabel}
\newcommand{\intExpEntLabel}{\textrm{int-exp-ent}}
\newcommand{\Ent}{H} % Entropy 
\newcommand{\weightDens}[1][\Ndesign]{\rho_{#1}}
\newcommand{\varInflation}{v}

% Basis function emulation. 
\newcommand{\basisVec}{\phi}
\newcommand{\basisWeight}{w}
\newcommand{\dimBasis}{R}
\newcommand{\idxBasis}{r}

% --------------------------------------------------------------------------------------------------
% Markov Chain Monte Carlo. 
% --------------------------------------------------------------------------------------------------

% Generic MCMC commands. 
\newcommand{\propDens}{q} % Proposal density
\newcommand{\propDist}{Q} % Proposal distribution (measure)
\newcommand{\propPar}{\tilde{\Par}} % Proposal density
\newcommand{\llikSamp}{\ell} 
\newcommand{\llikSampProp}{\tilde{\ell}} 
\newcommand{\llikSampDist}{\nu}
\newcommand{\CovProp}{\mathbf{C}} % Proposal covariance matrix
\newcommand{\sampleIndex}{k} % Generic index for samples. 
\newcommand{\NSample}{K} % Generic variable for algorithms that draw a set of samples (e.g., MCMC)
\newcommand{\indexSample}[2][\sampleIndex]{{#2}_{#1}}
\newcommand{\NMCMC}{\NSample} % Proposal covariance matrix
\newcommand{\accProbMH}{\alpha} % Metropolis-Hastings acceptance probability
\newcommand{\avgAccProbMH}{\overline{a}}
\newcommand{\accProbRatio}{r}
\newcommand{\likRatio}{\lambda}
\newcommand{\MarkovKernel}{P}
\newcommand{\mcmcIndex}{\sampleIndex}
% \newcommand{\indexMCMC}[2][\mcmcIndex]{{#2}_{#1}}
\newcommand{\indexMCMC}[2][\mcmcIndex]{\indexSample[{#1}]{#2}}
\newcommand{\AR}{\mathsf{A}} % Metropolis-Hastings accept-reject step

% Preconditioned Crank-Nicholson
\newcommand{\pcnCor}{\rho}

% --------------------------------------------------------------------------------------------------
% Dynamical Models. 
% --------------------------------------------------------------------------------------------------

\newcommand{\Time}{t}
\newcommand{\state}{x}
\newcommand{\odeRHS}{F}
\newcommand{\timeStart}{\Time_0}
\newcommand{\timeEnd}{\Time_1}
\newcommand{\stateIC}{\state_{\circ}}
\newcommand{\dimState}{S}
\newcommand{\stateIndex}{s}
\newcommand{\indexState}[2][\stateIndex]{{#2}^{({#1})}}
\newcommand{\solutionOp}{\mathcal{S}}
\newcommand{\obsOp}{\mathcal{H}}
\newcommand{\timeIndex}{\mcmcIndex}
\newcommand{\indexTime}[2][\timeIndex]{{#2}_{#1}}
\newcommand{\NTimeStep}{\NSample}
\newcommand{\forcing}{w}

% --------------------------------------------------------------------------------------------------
% Very Simple Ecosystem Model (VSEM)
% --------------------------------------------------------------------------------------------------

\newcommand{\stateV}{\state_{\textrm{v}}}
\newcommand{\stateR}{\state_{\textrm{r}}}
\newcommand{\stateS}{\state_{\textrm{s}}}
\newcommand{\dstateV}{\dot{\state}_{\textrm{v}}}
\newcommand{\dstateR}{\dot{\state}_{\textrm{r}}}
\newcommand{\dstateS}{\dot{\state}_{\textrm{s}}}
\newcommand{\NPP}{\textrm{NPP}}
\newcommand{\GPP}{\textrm{GPP}}
\newcommand{\alphaV}{\alpha_{\textrm{v}}}
\newcommand{\tauV}{\tau_{\textrm{v}}}
\newcommand{\tauR}{\tau_{\textrm{r}}}
\newcommand{\tauS}{\tau_{\textrm{s}}}
\newcommand{\LUE}{\ell}
\newcommand{\LAR}{r}
\newcommand{\KEXT}{\kappa}
\newcommand{\fracRespiration}{\gamma}
\newcommand{\LAI}{\textrm{LAI}}









\newcommand{\bphi}{\boldsymbol{\phi}}

\setlist{topsep=1ex,parsep=1ex,itemsep=0ex}
\setlist[1]{leftmargin=\parindent}
\setlist[enumerate,1]{label=\arabic*.,ref=\arabic*}
\setlist[enumerate,2]{label=(\alph*),ref=(\alph*)}

% For embedding images
\graphicspath{{./figures/}}

% Specifically for paper formatting 
\renewcommand{\baselinestretch}{1.2} % Spaces manuscript for easy reading

% Formatting definitions, propositions, etc. 
\newtheorem{definition}{Definition}
\newtheorem{prop}{Proposition}
\newtheorem{lemma}{Lemma}
\newtheorem{thm}{Theorem}
\newtheorem{corollary}{Corollary}

% Ensure cleverref knows how to handle citing of multiple subsubsections
\crefname{subsubsection}{section}{sections}
\Crefname{subsubsection}{Section}{Sections}

% Title and author
\title{Uncertainty Propagation and Active Learning for Surrogate-Based Bayesian Inference}
\author{Andrew Roberts}

\begin{document}

\maketitle

Simulation-based computer models are key tools for studying complex systems within 
the physical, biological, and engineering sciences. 
Such models often have uncertain parameters that must be estimated from data. 
Bayesian methods are commonly employed
to quantify uncertainty in these parameter calibration tasks.
However, standard Bayesian inference algorithms such as Markov chain Monte Carlo (MCMC) 
are hindered by the computational cost of the simulation model.
A popular approach to deal with this issue is to train a surrogate model 
(i.e., emulator) as a computationally thrifty approximation of the expensive computer code.
Surrogates have seen widespread use in Bayesian inference pipelines across a variety 
of applications [\todo: cite]. 

Despite significant advances in surrogate modeling, fitting a 
highly accurate emulator under a limited computational budget remains a challenging task.
In computationally-intensive applications, the surrogate error can be non-negligible. 
This error is propagated when the emulator is used to infer the values of the calibration 
parameters, potentially leading to overconfident results with miscalibrated uncertainty 
estimates in downstream applications \citep{BurknerSurrogate}.
It is thus crucial to acknowledge and propagate this 
additional source of uncertainty in surrogate-based Bayesian workflows.
Probabilistic surrogates such as Gaussian processes \citep{gpML,gramacy2020surrogates} 
and probabilistic neural networks \citep{deepEnsembles,BayesOptNN} provide a notion of 
predictive uncertainty that can utilized to this end
 \citep{reviewPaper,BilionisBayesSurrogates,BurknerSurrogate,CES,FerEmulation}.
 Despite this, applications of surrogate-based Bayesian inference routinely ignore 
 surrogate uncertainty [\todo], or propagate it in ad hoc ways [\todo: cite].
 
Typical surrogate-modeling workflows employ a two-step procedure
in which an emulator is fit to data generated from a computer model in the first stage, 
and then deployed in a second stage to accelerate inference for the computer 
model parameters \citep{modularization,BurknerTwoStep,BurknerSurrogate}. 
In the first step, emulators are trained to approximate a particular target quantity, 
often a summary of the computer model outputs 
(e.g., \citet{hydrologicalModel,idealizedGCM,Surer2023sequential}), the log-likelihood
function (e.g., \citet{VehtariParallelGP,FATES_CES,trainDynamics}), or the unnormalized 
log-posterior density (e.g., \citet{emPostDens,gp_surrogates_random_exploration,Kandasamy_2017}).
In the second step, the emulator 
is inserted in place of the target quantity to enable the application of Bayesian inference  
algorithms without requiring further queries to the expensive simulator.

This modular framework 
has several practical benefits \citep{modularization,PlummerCut}, but leaves open the 
question as to the ``correct'' approach for propagating surrogate uncertainty within the 
posterior approximation in the second stage. A variety of uncertainty-aware posterior 
approximations have been proposed, but little guidance exists on choosing a particular 
method \citep{reviewPaper,BilionisBayesSurrogates,StuartTeck1,VehtariParallelGP,
BurknerSurrogate,BurknerTwoStep,FerEmulation}. Moreover, previous research 
tends to focus on a particular surrogate model (e.g., Gaussian process) and target 
(e.g., the log-likelihood), limiting the ability to understand how a particular 
uncertainty propagation method may perform under different modeling setups. 
 
In addition to improving our conceptual understanding of competing uncertainty 
propagation methods, the development of practical, general-purpose computational 
algorithms are needed to support generic workflows for modular surrogate-based 
Bayesian inference. A wide body of work has sidestepped computational difficulties
by focusing on special cases that allow for tractable surrogate approximations of the 
likelihood function 
\citep{StuartTeck1,GP_PDE_priors,hydrologicalModel,hydrologicalModel2,
Surer2023sequential,VillaniAdaptiveGP,weightedIVAR,idealizedGCM,CES,
FATES_CES,VehtariParallelGP,quantileApprox}. A closed-form likelihood approximation
enables the use of standard MCMC software. However, such approaches typically require 
particular distributional assumptions on the emulator predictive distribution and the 
likelihood function. \citep{garegnani2021NoisyMCMC} studies the application of 
pseudo-marginal and ``noisy'' MCMC algorithms, which do not rely on such distributional
assumptions. \citep{BurknerSurrogate} propose a generic probabilistic programming workflow,
including MCMC schemes for a variety of different uncertainty propagation methods. 

[\todo: need to motivate our work here; these two papers don't directly apply to the GP setting]

\paragraph{Contributions.}
As summarized above, a wide variety of methods have been proposed for 
uncertainty-aware posterior approximation in surrogate-based Bayesian inference. 
In this paper, we provide new insights into when and why the performance of different 
methods can deviate. In particular, we investigate the implications of propagating
uncertainty through the derivation of a pointwise likelihood approximation, as compared 
to aggregating samples from an ensemble of posterior approximations. We demonstrate 
that the extent to which these two methods agree depends crucially on the emulator 
target and on the tails of the emulator predictive distribution. 

[\todo: paragraph on MCMC] [\todo: paragraph on practical considerations/model checking]

These algorithms require only
slight adjustments to standard MCMC implementations, and are generally applicable to any 
surrogate with a predictive distribution from which samples can be drawn.

% Surrogates for Bayesian Inverse Problems
\section{Surrogates for Bayesian Inverse Problems}

\subsection{Bayesian Inference Setting}
We consider the general goal of estimating parameters $\Par \in \parSpace \subseteq \R^{\dimPar}$ given 
observations $\obs \in \obsSpace \subseteq \R^{\dimObs}$ within a Bayesian framework.
Assuming a prior density $\priorDens(\Par)$ and likelihood $p(\obs \given \Par)$, 
we seek to characterize the posterior distribution 
\begin{align}
&\postDensNorm(\Par) \Def p(\Par \given \obs) = \frac{1}{\normCst} \priorDens(\Par) p(\obs \given \Par), 
&&\normCst = \int_{\parSpace} \priorDens(\Par) p(\obs \given \Par) d\Par, \label{eq:post_dens_generic}
\end{align}
where $\normCst$ is an (intractable) normalizing constant. 
Let $\postDens(\Par) \Def \priorDens(\Par) p(\obs \given \Par)$ denote the unnormalized posterior 
density, which plays a central role in our development. 
We assume that pointwise evaluations of  $\postDens(\Par)$ can be computed, but at significant 
computational expense. This renders standard iterative optimization and sampling algorithms 
(e.g., MCMC) infeasible, as each iteration requires a new density evaluation $\postDens(\Par)$.

\subsection{Bayesian Inverse Problems}
The challenge posed by computationally expensive density evaluations $\postDens(\Par)$ commonly 
arises in the Bayesian approach to inverse problems \citep{Stuart_BIP}. In this setting, 
the likelihood often takes the form $\obs = \fwd(\Par) + \noise$ for some forward model
$\fwd: \parSpace \to \obsSpace$. For a concrete example, we consider the problem of estimating the 
parameters in a system of ordinary differential equations (ODEs)
\begin{align}
\frac{d}{d\Time} \state(\Time, \Par) &= \odeRHS(\state(\Time, \Par), \Par), &&x(\timeStart) = \stateIC, \label{ode_ivp}
\end{align}
where the dynamics depend on parameters $\Par$. Each value for $\Par$ implies a different solution trajectory
$[\state(\Time, \Par)]_{\timeStart \leq \Time \leq \timeEnd}$, which we encode by the
map $\solutionOp: \Par \mapsto [\state(\Time, \Par)]_{\timeStart \leq \Time \leq \timeEnd.}$. The goal is then 
to identify the parameters that yield trajectories in agreement with observed data 
$\obs$, which is assumed to be some noise-corrupted function $\obsOp$ of the true trajectory. Thus, the 
likelihood is of the form 
\begin{align}
&\obs = \fwd(\Par) + \noise, &&\fwd \Def \obsOp \circ \solutionOp. \label{eq:additive-noise}
\end{align}
In practice, the ODE is solved numerically so $\solutionOp$ represents the map induced by a numerical 
solver. Therefore, in this setting the computational cost of computing $\postDens(\Par)$ stems from the 
dependence of the likelihood on $\fwd(\Par)$, and in particular on the solver $\solutionOp(\Par)$.

\subsection{Surrogates Targets for Bayesian Inference} \label{sec:surrogates-Bayes}
Given the cost of computing $\postDens(\Par)$, we seek to approximate
the posterior using a small set of queries to the posterior density. 
Surrogate models (i.e., emulators) leverage this limited information by 
learning a map which yields predictions of $\postDens(\Par)$ at new values 
of $\Par$. Typically, the emulator does not learn the map
$\Par \mapsto \postDens(\Par)$ directly, but rather induces an approximation
of $\postDens$ by approximating some underlying target map 
$\Par \mapsto \target(\Par)$. Different target maps may be emulated in order to
 accelerate Bayesian inference, so long as $\target(\Par)$ is sufficient to compute 
 $\postDens(\Par)$ without requiring additional simulator runs. We write 
 $\postDens(\Par; \target)$ to indicate the dependence of the posterior density 
 on $\target$, with the assumption that $\postDens(\Par; \target)$ is relatively 
 quick to compute once the expensive computation $\target(\Par)$ is
 obtained. In addition, let $\lik(\target(\Par); \obs)$ denote the likelihood parameterized
 as a function of the target quantity. 
 
Consider a computational budget of $\Ndesign$ evaluations of $\target$, where 
$\Ndesign$ is typically much smaller than that required by standard posterior
inference algorithms. Surrogate modeling offers a solution to this problem 
by using the queries $\{(\Par_n, \target(\Par_n))\}_{n=1}^{\Ndesign}$ to 
fit a regression or interpolation model $\targetEm$, such that $\targetEm(\Par)$
provides a prediction of $\target(\Par)$ at new values of $\Par$.
Replacing the true target $\target$ with its emulator $\targetEm$ induces a 
surrogate posterior density $\postDens(\Par; \targetEm)$. In order to quantify
the uncertainty introduced via this approximation, we consider emulators 
that provide predictions in the form of a probability distribution; 
i.e., $\targetEm(\Par)$ is random and thus $\targetEm$ is a random function. 
We write $\emDist(\cdot \given \Par)$ and $\emDist$ to denote the distributions
of $\targetEm(\Par)$ and $\targetEm$, respectively. Since the emulator is random, 
$\postDens(\Par; \targetEm)$ is a univariate random variable and 
$\postDens(\cdot; \targetEm)$ a random 
function. We use the shorthand $\postEm$ to denote the former when the particular 
emulated quantity is not of consequence.

The manner in which the uncertainty in $\targetEm$ propagates to $\postEm$
depends on the target $\target$ as well as the form of the exact density 
$\postDens$. Below, we highlight two general choices of $\target$ common 
in the literature. This categorization has also 
been explored in \citet{StuartTeck1,StuartTeck2,GP_PDE_priors}. See
\citet{reviewPaper} for a discussion of practical considerations in choosing 
the emulator target. 

\subsubsection{Forward Model Emulation}
One natural approach is to target the underlying forward model 
$\Par \mapsto \fwd(\Par)$ (i.e., choosing $\target \Def \fwd$), a strategy 
we refer to as \textit{forward model emulation}.
This method consists of fitting a surrogate $\targetEm$ to the design 
$\{(\Par_n, \fwd(\Par_n))\}_{n=1}^{\Ndesign}$ and then using $\targetEm$
in place of $\fwd$. Much previous work has considered this strategy 
in the context of the additive noise model in \Cref{eq:additive-noise},
under the Gaussian assumption $\noise \sim \Gaussian(0, \likPar)$
\citep{StuartTeck1,GP_PDE_priors,hydrologicalModel,hydrologicalModel2,
Surer2023sequential,VillaniAdaptiveGP,weightedIVAR,idealizedGCM,CES}.
In this special case, the induced posterior density surrogate takes the form
\begin{align}
\postEm(\Par) &= \priorDens(\Par) \Gaussian(\obs \given \targetEm(\Par), \likPar). \label{eq:post-em-fwd-Gaussian}
\end{align}

\subsubsection{Log-Density Emulation}
Other lines of research have instead targeted the log-likelihood $\Par \mapsto \log \lik(\target(\Par); \obs)$
\citep{VehtariParallelGP,FATES_CES,trainDynamics,quantileApprox,
ActiveLearningMCMC,FerEmulation,StuartTeck1,random_fwd_models,
GP_PDE_priors,OakleyllikEm,JosephMinEnergy,AlawiehIterativeGP}
or the (unnormalized) log-posterior 
$\Par \mapsto \log \left\{\priorDens(\Par)\lik(\target(\Par); \obs)\right\}$
\citep{emPostDens,Kandasamy_2017,llikRBF,gp_surrogates_random_exploration,
landslideCalibration}.

We collectively refer to these strategies as \textit{log-density emulation}. In the 
log-likelihood case, an emulator $\targetEm$ is fit to a design 
$\{(\Par_n, \log \lik(\target(\Par_n); \obs)\}_{n=1}^{\Ndesign}$
and induces a posterior density surrogate 
\begin{align}
\postEm(\Par) &= \priorDens(\Par) \Exp{\targetEm(\Par)}. \label{eq:post-em-llik}
\end{align}
The log-posterior case is quite similar, except that the effect of the prior is also 
approximated by the emulator, so the induced posterior surrogate simply takes 
the form $\postEm(\Par) = \Exp{\targetEm(\Par)}$.

% Comparing Uncertainty Propagation Methods
\section{Comparing Uncertainty Propagation Methods} \label{sec:compare-unc-prop}
The second stage in the modular surrogate workflow consists of using the 
trained emulator $\targetEm$ to approximate the posterior $\postDensNorm$.
To avoid overconfident posterior inference, the uncertainty in $\targetEm$
should be propagated within this posterior approximation. Given the lack of
a unifying probabilistic model across the two stages, proper uncertainty 
quantification is not automatically given by standard Bayesian conditioning.
Consequently, many different uncertainty propagation methods have been 
proposed [\todo: cite]. This problem shares a close resemblance with other 
instances of modular inference, including data imputation [\todo: cite],
cut distributions [\todo: cite], and dimension reduction via active subspaces [\todo: cite].

Irrespective of the underlying target $\target$, the probabilistic emulator $\targetEm$ 
induces a random approximation of the posterior density
\begin{align}
\postNormEm(\Par) &\Def \frac{1}{\normCstEm} \postEm(\Par),
&&\normCstEm \Def \int_{\parSpace} \postEm(\Par) d\Par, \label{eq:random-post}
\end{align}
which is referred to as the ``sample'' approximation in \citet{StuartTeck1}. We can thus
view the challenge of uncertainty propagation as that of constructing a deterministic 
posterior approximation that summarizes the uncertainty encoded in $\postNormEm$.
Below we introduce two methods that have been proposed for this purpose. We generically
analyze when the behavior of the two methods tends to deviate, and show how these 
results translate to forward model and log-density emulators under tractable 
Gaussian assumptions.

\subsubsection{Expected Posterior}
A deterministic summary of $\postNormEm$ can be constructed via the expectation 
\begin{equation}
\postApproxEP(\Par) \Def \E_{\Ndesign}\left[\postNormEm(\Par) \right], \label{eq:ep-approx}
\end{equation}
with $\E_{\Ndesign}$ denoting the expectation with respect to the underlying emulator 
distribution $\emDist$.
We follow the terminology in \citet{BurknerSurrogate} and call $\postApproxEP$ the 
\textit{expected posterior (EP)}. The EP arises as the marginal conditional $\Par \given \obs$ under 
the joint probability model over $(\Par, \targetEm, \obs)$ defined generatively via
\begin{align}
\Par &\sim \priorDens \label{eq:ep-prob-model} \\
\targetTraj &\sim \emDist \nonumber \\
\obs \given \Par, \targetTraj &\sim \lik(\targetTraj(\Par); \cdot) \nonumber
\end{align}
In principle, $\postApproxEP$ can be directly sampled via \Cref{alg:ep} (see \Cref{sec:computation}
for a detailed discussion), implying that $\postApproxEP$ represents the aggregation of samples 
from an ensemble of posterior approximations, each corresponding to a single trajectory of $\targetEm$.

\begin{algorithm}
    \caption{Direct sampling from $\postApproxEP$}
    \label{alg:ep}
    \begin{algorithmic}[1]
    \Function{sampleEP}{$\llikEmRdm[\Ndesign]{\postDensNorm}, \NSample, M$}     
        \For{$\sampleIndex \gets 1, \dots, \NSample$} 
        		\State $\targetTraj^{(\sampleIndex)} \sim \emDist$ \Comment{Sample emulator trajectory}
		\State $\Par^{(\sampleIndex, 1)}, \dots, \Par^{(\sampleIndex, M)} \overset{iid}{\sim} \postDensNorm(\cdot; \targetTraj^{(\sampleIndex)})$ \Comment{Sample posterior given trajectory}
	\EndFor
	\State \Return $\{\Par^{(\sampleIndex, m)}\}_{1 \leq \sampleIndex \leq \NSample, \ 1 \leq m \leq M}$
	\EndFunction
    \end{algorithmic}
\end{algorithm}

Note that the quantity $\postNormEm(\Par)$ depends on the entire random function $\targetEm$ 
through its dependence on $\normCstEm$, as opposed to only depending on the prediction 
$\targetEm(\Par)$ at the single input $\Par$. Therefore, care must be taken in ensuring the 
expectation in \Cref{eq:ep-approx} is well-defined in the case that $\targetEm$ is an infinite-dimensional
random element (e.g., a GP). We assume throughout that $\targetEm$ is constructed such that 
trajectories of $\postNormEm$ are almost surely integrable, implying the sampling procedure
in \Cref{alg:ep} is well-defined. See 
\citet{StuartTeck1,StuartTeck2,random_fwd_models,garegnani2021NoisyMCMC} for additional
technical details. 

Beyond surrogate uncertainty propagation, the EP appears as the ``cut posterior'' in the generic
modular Bayesian inference setting \citep{PlummerCut}, and the sampling procedure in 
\Cref{alg:ep} is closely related to multiple imputation algorithms used to handle missing data [\todo: cite].
The notion of aggregating multiple posterior distributions is also used in contexts other than
modular inference, including for robustness to model misspecification \citep{BayesBag,BayesBag2}. 
Despite these similarities, our present setting features a few key differentiating characteristics.
First, the ``stage one'' quantity $\targetEm$ is a random function. The potentially infinite-dimensional 
nature of $\targetEm$ presents additional practical difficulties not typically present in these other 
settings, the consequences of which are discussed in \Cref{sec:computation}.
In addition, a conceptual distinction is that we do not assume $\targetEm$ is necessarily derived 
using Bayesian methods, and thus do not view this two-stage framework as an approximation to 
an overarching Bayesian graphical model. The predictive distribution of the emulator may arise 
via non-Bayesian methodology, such as deep ensembles \citep{deepEnsembles} or 
epistemic neural networks \citep{epistemicNN,BayesOptEpistemicNN}. 

\subsubsection{Expected Likelihood}
As noted above, the presence of $\normCstEm$ in the expectation in \Cref{eq:ep-approx}
can present computational challenges. Citing these difficulties, previous research has 
eschewed the EP in favor of approximations of the \textit{unnormalized}
density surrogate $\postEm$ \citep{StuartTeck1,StuartTeck2,VehtariParallelGP}. Unlike 
$\postNormEm(\Par)$, $\postEm(\Par)$ depends only on the single-point prediction 
$\targetEm(\Par)$ and thus can be approximated in a pointwise fashion. 
Summarizing $\postEm(\Par)$ with its expectation, we obtain the \textit{expected likelihood (EL)} approximation
\begin{equation}
\postApproxMarg(\Par) \Def 
\frac{\E_{\Ndesign}\left[\postEm(\Par) \right]}{\int_{\parSpace} \E_{\Ndesign}\left[\postEm(\Par) \right] d\Par}
= \frac{\E_{\Ndesign}\left[\postEm(\Par) \right]}{\E_{\Ndesign}[\normCstEm]}, \label{eq:post-approx-EL} 
\end{equation}
a term again borrowed from \cite{BurknerSurrogate}. The equality in \Cref{eq:post-approx-EL} follows from
changing the order of integration, courtesy of Tonelli's theorem \citep{StuartTeck1}. 

The EL approximation is the marginal conditional $\Par \given \obs$ under the joint model on 
$(\Par, \gamma, \obs)$ defined generatively as
\begin{align}
\Par &\sim \priorDens \label{eq:el-prob-model} \\
\gamma \given \Par &\sim \emDist(\cdot \given \Par) \nonumber \\
\obs \given \gamma &\sim \lik(\gamma; \cdot). \nonumber
\end{align}
This joint model is proposed as the basis for surrogate uncertainty propagation in 
\citet{BilionisBayesSurrogates}, and is also used as justification for the EL approximation
in \citet{SinsbeckNowak}. \citet{StuartTeck2,CES} also note this extended parameter space 
perspective in the Gaussian setting of \Cref{eq:post-em-fwd-Gaussian}. The EL approximation
has been widely used in special cases that admit a closed-form expression for 
$\E_{\Ndesign}\left[\postEm(\Par) \right]$. This primarily includes the use of GP forward
model emulators with Gaussian likelihoods 
\citep{weightedIVAR,StuartTeck2,GP_PDE_priors,CES,idealizedGCM,
villani2024posteriorsamplingadaptivegaussian,hydrologicalModel,hydrologicalModel2},
as well as GP log-density emulators 
\citep{VehtariParallelGP,StuartTeck1,StuartTeck2,GP_PDE_priors, random_fwd_models,TeckHyperpar}. 
We analyze these special cases in \Cref{sec:ldens-Gaussian,sec:fwd-Gaussian}.

\paragraph{TODO, Need to address: } The formulation in \Cref{eq:el-prob-model} really only 
seems to make conceptual sense in the forward model emulation case; e.g., 
$\lik(\gamma; \cdot) = \Gaussian(\obs \given \gamma, \likPar)$ truly does define a conditional 
distribution $p(\obs \given \gamma)$. However, for a log-likelihood emulator, $\gamma$ is a log-likelihood
value, so  $\lik(\gamma; \cdot) = \Exp{\gamma}$. We could potentially try to address this via something
like $p(\obs \given \gamma) \Def \delta_{e^\obs}(e^\gamma)$, but this doesn't really make sense.
Under the true model, there is sampling variability in $\obs$ encoded by the true likelihood 
$p(\obs \given \Par)$, and uncertainty in this sampling variability is contained in $\funcEm$.

I think for computational convenience this sort of extended parameter space idea can still be useful
in deriving algorithms. But the likelihood really isn't a conditional distribution, so this would represent 
some sort of generalized posterior. 

Perhaps it is worth adding discussion of these subtleties with regard to the goal of general-purpose 
uncertainty propagation methods. While the extended parameter space model seems to mainly make
sense for forward model emulation, all of these approaches can be unified through the random measure
view. However, are the same sort of uncertainty propagation methods reasonable for all of these approaches,
or is there something fundamentally different about log-density emulators that implies that the two cases 
should be addressed separately?


\subsection{Comparison of EP and EL}
While there are many instances of either the EP or EL being employed as the chosen uncertainty
propagation method, there has been little work on comparing the performance of the two approaches.
One exception is \citet{BurknerSurrogate}, who conduct an empirical comparison of these methods
in several case studies. 
\footnote{Technically, \citet{BurknerSurrogate} empirically test approximations to both EP and EL, derived
by replacing the expectations in \Cref{eq:ep-approx,eq:post-approx-EL} with sample average approximations.}
The authors ultimately endorse both approximations, concluding that they produce ``non-equivalent''
but ``very similar'' inference in their applications. However, the experiments are limited to relatively simple 
models involving polynomial-based forward model emulators. \citet{garegnani2021NoisyMCMC} also
consider both approaches, though their purpose is to study the performance of sampling algorithms, rather
than compare uncertainty propagation methods.

Note that the difference between $\postApproxEP(\Par) = \E_{\Ndesign}[\postEm(\Par)/\normCstEm]$ and 
$\postApproxNormMarg(\Par) = \E_{\Ndesign}\left[\postEm(\Par) \right]/\E_{\Ndesign}[\normCstEm]$ 
stems from the fact that the expectation of a ratio does not 
in general equal to the ratio of the expectations. An application of the Delta method [\todo: add in appendix] 
yields the approximation
\begin{align}
\postApproxEP(\Par)
&\approx \postApproxNormMarg(\Par)[1 + \cv(\normCstEm) \tau_{\Ndesign}(\Par)] \label{eq:delta-method} \\
\tau_{\Ndesign}(\Par) &\Def \cv(\normCstEm) - \Cor(\postEm(\Par), \normCstEm) \cv(\postEm(\Par)), \nonumber
\end{align}
where $\cv(\psi) \Def \sqrt{\Var(\psi)}/\E[\psi]$ denotes the coefficient of variation of a random 
variable $\psi$ and $\Cor(\psi, \psi^\prime)$ is the correlation coefficient between $\psi$
and $\psi^\prime$. According to this second order approximation, when 
$\cv(\normCstEm) \tau_{\Ndesign}(\Par)$ is positive then $\postApproxEP(\Par) > \postApproxNormMarg(\Par)$,
and vice versa. Since these densities must integrate to
one over $\parSpace$, a positive discrepancy in one region implies a negative discrepancy in another.
Based on \Cref{eq:delta-method}, we expect the two approximations to roughly agree when variability 
in $\normCstEm$ is small, which happens when $\targetEm$ contains little uncertainty, or 
$\postDens(\cdot; \target)$ is not sensitive to perturbations to $\target$. Alternatively, we expect 
agreement when $\tau_{\Ndesign}(\Par) \approx 0$ for most $\Par$. This happens when
$\Cor(\postEm(\Par), \normCstEm) \approx \cv(\normCstEm) / \cv(\postEm(\Par))$; i.e., the ratio of the 
variability in $\normCstEm$ over the variability in $\postEm(\Par)$ is well-approximated by the 
correlation between these two quantities. Such a linear relationship will generally not hold when 
$\postEm(\Par)$ is heavy-tailed. In this case the ratio of means
$\E_{\Ndesign}\left[\postEm(\Par) \right]/\E_{\Ndesign}[\normCstEm]$ will typically exceed the average
realization of $\postEm(\Par)/\normCstEm$. For the latter, large realizations of $\postEm(\Par)$
are typically correlated with large values of $\normCstEm$, which moderates the magnitude of 
$\postEm(\Par)/\normCstEm$. In the extreme case where $\postEm(\Par)$ has a very heavy 
upper tail, then $\Cor(\postEm(\Par), \normCstEm) \cv(\postEm(\Par)) \gg \cv(\normCstEm)$ and
consequently $\postApproxNormMarg(\Par) \gg \postApproxEP(\Par)$. In other words, the 
EL approximation is susceptible to concentrating in regions of $\parSpace$ where the 
surrogate $\postEm(\Par)$ is highly uncertain---particularly when $\postEm(\Par)$ has a 
heavy upper tail. This sensitivity may be mitigated in cases where the emulator predictive
distribution is bounded from above. We explore these ideas in the context of GP emulators 
below. 

\subsubsection{Gaussian Log-Density Emulation Setting} \label{sec:ldens-Gaussian}
We now consider the particular example where $\postEm(\Par) = \Exp{\targetEm(\Par)}$
and $\targetEm \sim \GP(\gpMean, \gpKer)$ is a GP emulator; that is, 
$\E[\targetEm(\Par)] = \gpMean(\Par)$ and 
$\Cov[\targetEm(\Par),\targetEm(\Par^\prime)] = \gpKer(\Par, \Par^\prime)$ for all 
$\Par, \Par^\prime \in \parSpace$. We also make use of the shorthand 
$\gpKer(\Par) \Def \gpKer(\Par, \Par)$. This setup encompasses both the log-likelihood
and log-posterior emulation settings via suitable definitions of $\gpMean$ and $\gpKer$.
GP log-density emulators have been used in various studies 
\citep{VehtariParallelGP,FATES_CES,trainDynamics,quantileApprox,ActiveLearningMCMC,FerEmulation,
StuartTeck1,random_fwd_models,GP_PDE_priors,OakleyllikEm,JosephMinEnergy,AlawiehIterativeGP}.

In this setting, the pointwise distribution of the induced posterior surrogate is given by the log-normal 
distribution $\postEm(\Par) \sim \LN(\gpMean(\Par), \gpKer(\Par))$, implying that
\begin{equation}
\postApproxNormMarg(\Par) \propto \Exp{\gpMean(\Par) + \frac{1}{2}\gpKer(\Par)}.
\label{eq:EL-Gaussian-fwd}
\end{equation} 
Using the known form of the coefficient of variation of a log-normal variable, we also see that
$\cv(\postEm(\Par)) = \sqrt{\Exp{\gpKer(\Par)} - 1}$. Even at relatively small values for the 
predictive variance (say, $\gpKer(\Par) > 4$), this quantity scales like $\Exp{\gpKer(\Par)/2}$
and thus blows up quickly as uncertainty increases. In very uncertain regions, we thus expect
$\tau_{\Ndesign}(\Par) \ll 0$ and $\postApproxNormMarg(\Par) \gg \postApproxEP(\Par)$.
To better understand how the uncertainty propagates, we compare to the ``plug-in mean''
 baseline approximation $\postApproxNormMean(\Par) \propto \Exp{\gpMean(\Par)}$, 
 which ignores the emulator uncertainty. Consider the uncertainty inflation factor
 \begin{equation}
 \lambda(\Par, \Par^\prime) 
 \Def \frac{\postApproxNormMarg(\Par^\prime)/\postApproxNormMarg(\Par)}{\postApproxNormMean(\Par^\prime)/\postApproxNormMean(\Par)}
 = \Exp{\frac{1}{2}\left[\gpKer(\Par^\prime) - \gpKer(\Par) \right]},
 \label{eq:unc-infl-factor}
 \end{equation}
which measures the EL relative weighting between the points $(\Par, \Par^\prime)$, relative to the 
weighting assigned by the plug-in mean approximation. We conclude from \Cref{eq:unc-infl-factor}
that, if $\gpMean(\Par^\prime) = \gpMean(\Par)$, then $\postApproxNormMarg(\Par^\prime) > \postApproxNormMarg(\Par)$
when $\gpKer(\Par^\prime) > \gpKer(\Par)$. In other words, the EL approximation inflates the posterior
in accordance with the magnitude of the emulator uncertainty at a particular point. 
This is not true in general; see \Cref{sec:fwd-Gaussian}. Moreover, we see from \Cref{eq:unc-infl-factor} that
$\postApproxNormMarg(\Par^\prime) \gg \postApproxNormMarg(\Par)$ even when 
$\gpKer(\Par^\prime) - \gpKer(\Par)$ is only modestly large, owing to the exponential scaling. 
For example, when the standard deviation of $\targetEm(\Par^\prime)$ is twice that of $\targetEm(\Par)$,
then $\lambda(\Par, \Par^\prime) = \Exp{\frac{3}{2}\gpKer(\Par)}$. If $\gpKer(\Par) = 4$, this means
$\Exp{\frac{3}{2}\gpKer(\Par)} \approx 400$ -- the ratio $\postApproxNormMarg(\Par^\prime)/\postApproxNormMarg(\Par)$
is $400$ times larger than $\postApproxNormMean(\Par^\prime)/\postApproxNormMean(\Par)$.

We conclude that great care must be taken when propagating uncertainty using a log-density emulator. 
A seemingly small level of surrogate uncertainty on the log-density scale can easily dominate the posterior
approximation. Our analysis above demonstrates that the EL approximation is especially susceptible to 
this issue. In numerical experiments we find that the EP approximation is more robust in this respect, 
but not immune [\todo: add illustrative plots]. \citet{VehtariParallelGP} are well-aware of this concern, 
noting that the expectation of $\postEm(\Par)$ often provides a poor summary of the predictive distribution, 
owing to the heavy log-normal tail. Their solution is to instead define a posterior approximation via a pointwise 
quantile 
$\postNormEm^{\textrm{mean}}(\Par) \propto \mathrm{Quantile}_{\quantileProb}(\postEm(\Par)) = \Exp{\gpMean(\Par) + \GaussianCDF^{-1}(\quantileProb) \sqrt{\gpKer(\Par)}}$, where $\GaussianCDF$ denotes the standard Gaussian distribution function.
This quantile approximation is also considered in \citet{quantileApprox,FATES_CES}. While more robust, the quantile 
approximation is still quite sensitive to uncertainty in the log-density emulator. The exception is when $\alpha = 1/2$, in which 
case the quantile approximation reduces to the plug-in mean approximation. This is the choice used in the experiments
conducted by \citet{VehtariParallelGP}. 

\subsubsection{Gaussian Forward Model Emulation Setting} \label{sec:fwd-Gaussian} 
Consider the model
\begin{align}
&\obs = \fwd(\Par) + \noise, &&\noise \sim \Gaussian(0, \likPar)
\end{align}
with prior $\Par \sim \priorDens$ independent of $\noise$. We consider the case 
where $\fwd$ is replaced with a (potentially multi-output) GP emulator 
$\targetEm \sim \GP(\gpMean, \gpKer)$. This setup has been considered frequently in various 
applications \citep{Surer2023sequential,weightedIVAR,StuartTeck2,GP_PDE_priors,CES,
idealizedGCM,villani2024posteriorsamplingadaptivegaussian,hydrologicalModel,hydrologicalModel2}.

The unnormalized posterior surrogate takes the form 
$\postEm(\Par) = \priorDens(\Par)\lik(\funcEm(\Par); \obs) =  \priorDens(\Par)\Gaussian(\obs \mid \targetEm(\Par), \likPar)$, implying
\begin{equation}
\postApproxNormMarg(\Par) \propto \priorDens(\Par) \Gaussian(\obs \mid \gpMean(\Par), \likPar + \gpKer(\Par)),
\label{eq:EL-Gaussian-fwd}
\end{equation}
following from the formula for the convolution of two Gaussians [\todo: add to appendix].
A key observation in this setting is that $\postEm(\Par) \in [0, B_{\likPar}]$, where 
$B_{\likPar} \Def \sup_{u \in \parSpace} \priorDens(\Par) / \det(2\pi\likPar)^{1/2}$.
[\todo: need to finish this thought. The conjecture here is that the EL in this setting is 
more robust in this setting due to the bound. May be able to show that $\cv$ is upper
bounded--issue is the possibility of expectation approaching 0].

Unlike in the Gaussian log-density emulation setting, in this case the EL does does simply 
inflate the posterior in accordance with the magnitude of emulator uncertainty at different points.
In fact, it is possible for an increase in $\gpKer(\Par)$ to deflate the value of $\postApproxNormMarg(\Par)$.
To investigate this, we again consider the plug-in mean approximation
$\postApproxNormMean(\Par) \propto \priorDens(\Par) \Gaussian(\obs \mid \gpMean(\Par), \likPar)$,
which ignores the emulator uncertainty. The ratio 
$\postApproxNormMarg(\Par)/\postApproxNormMean(\Par)$ can be above 
or below one, depending on the distance between $\obs$ and $\gpMean(\Par)$. In particular,
if $\obs$ and $\gpMean(\Par)$ are close, then increasing $\gpKer(\Par)$ actually has the effect
of deflating the EL density at $\Par$. This is due to the fact that the Gaussian density is 
upper bounded, and thus near the upper bound the expectation has nowhere to go but down.
[\todo: in cases, where the forward model is quite imperfect we would expect $\obs$ and $\gpMean(\Par)$
to typically be quite far apart. In such settings, can the EL approximation exhibit similar sensitivity to 
emulator uncertainty as in the log-density setting?]
 
% Computation for the Posterior Approximations
\section{Computation for the Posterior Approximations} \label{sec:computation}
Despite the widespread use of surrogates in various application domains, 
surrogate uncertainty is often ignored or propagated in an ad hoc manner.
This can be explained in part due to poor conceptual understanding 
of the consequences of different uncertainty propagation methods
(\Cref{sec:compare-unc-prop}), but also to a lack of general-purpose software
for computation. Previous work has often relied on specific assumptions that
enable closed-form computations, with algorithms geared towards
particular emulators or likelihoods [\todo: cite]. Recently, \citep{BurknerSurrogate}
proposed a generic MCMC pipeline rooted in standard probabilistic programming
workflows. While a step in the right direction, a potential drawback of their approach
is that their MCMC methods only approximately target the desired distributions 
due to their use of fixed Monte Carlo estimates (i.e., sample average approximations).
Another limitation is that their approach does not directly accommodate 
infinite-dimensional surrogates such as GPs, owing to the requirement of 
sampling trajectories of $\funcEm$.

We aim to address these challenges, focusing primarily on MCMC schemes for the EL
and EP approximations that are applicable to generic probabilistic surrogate models 
and likelihood functions. While general-purpose computation for the EL approximation 
has been studied \citep{garegnani2021NoisyMCMC}, existing algorithms for EP-based 
inference have been deemed impractical for GP emulators \citep{VehtariParallelGP,StuartTeck2}.
We present a new MCMC method that is applicable to the GP setting, and is
shown to provide a close approximation of the EP distribution in various experiments.

Prior to presenting this algorithm, we provide a thorough review of MCMC methods
targeting the EL and EP distributions, or approximations to these distributions. 
We emphasize connections between pseudo-marginal MCMC \citep{pseudoMarginalMCMC}, 
``noisy'' MCMC \citep{noisyMCMC}, and inference schemes for cut 
posteriors \citep{PlummerCut}. We also highlight an ``approximate computation'' 
perspective to the uncertainty propagation question. Typically, inference proceeds
by first defining a target distribution and then designing an MCMC scheme that
converges to this distribution. Alternatively, one might skip the first step and instead 
alter a standard MCMC scheme by strategically injecting noise within the 
algorithm to account for the surrogate uncertainty. Assuming the altered algorithm 
still converges to a well-defined posterior distribution, the implicitly-defined posterior 
approximation can be interrogated only by running the algorithm.
We show that intuitively following this approach recovers several existing algorithms, 
but that care must be taken as certain design choices can arbitrarily change the 
degree to which uncertainty is propagated.

\paragraph{Notation.} Throughout this section, we define MCMC algorithms
based on the steps performed for a single iteration of computation. For any 
state variable $\theta$, we write $\theta$ and $\theta^\prime$ for the current
and updated state, respectively. We also use the function defined by 
\Cref{alg:acc-rej} to succinctly represent the Metropolis-Hastings
accept-reject step.

\begin{algorithm}[H]
    \caption{Metropolis-Hastings Accept-Reject Step}
    \label{alg:acc-rej}
    \begin{algorithmic}[1]
    \Function{$\AR$}{$\Par$, $\propPar$, $\targetTraj$, $\targetTrajProp$}     
        \For{$\sampleIndex \gets 1, \dots, \NSample$} 
        		\State $\targetTraj^{(\sampleIndex)} \sim \mathrm{law}(\targetEm)$ \Comment{Sample emulator trajectory}
		\State $\Par^{(\sampleIndex, 1)}, \dots, \Par^{(\sampleIndex, M)} \overset{iid}{\sim} \postDensNorm(\cdot; \targetTraj^{(\sampleIndex)})$ \Comment{Sample posterior given trajectory}
	\EndFor
	\State \Return $\{\Par^{(\sampleIndex, m)}\}_{1 \leq \sampleIndex \leq \NSample, \ 1 \leq m \leq M}$
	\EndFunction
    \end{algorithmic}
\end{algorithm}

\subsection{Expected Likelihood}
We start by summarizing algorithms proposed for EL-based inference. Recall the target distribution in this 
case is $\postApproxNormMarg(\Par) \propto \E_{\Ndesign}[\postEm(\Par)]$. If the expectation in this 
expression is available in closed form, then off-the-shelf samplers can be applied. This is the case in the 
Gaussian settings considered in \Cref{sec:ldens-Gaussian,sec:fwd-Gaussian}. However, alternative 
algorithms are required to generalize to more complicated emulators and likelihoods. 
As a starting point for general-purpose algorithms, we might consider the fact that the extended model 
in \Cref{eq:el-prob-model} admits $\postApproxNormMarg$ as a marginal conditional. We can thus 
consider an MCMC scheme targeting the joint density
\begin{equation}
p(\Par, \gamma \given \obs) \propto \priorDens(\Par) \lik(\gamma; \obs) \emDist(\gamma \given \Par). 
\end{equation}
If we consider a joint Metropolis-Hastings update for $(\Par, \gamma)$ using a proposal of the form
\begin{equation}
\propDens(\tilde{\Par}, \tilde{\gamma} \given \Par, \gamma)
\Def \propDens_{\Par}(\tilde{\Par} \given \Par) \emDist(\tilde{\gamma} \given \tilde{\Par}),
\end{equation}
then the standard Metropolis-Hastings acceptance probability assumes the form
\begin{equation}
\accProbMH(\tilde{\Par}, \tilde{\gamma} \given \Par, \gamma)
= \min\left\{1, \frac{\priorDens(\propPar) \lik(\tilde{\gamma}; \obs)\propDens_{\Par}(\Par \given \tilde{\Par})}{\priorDens(\Par)\lik(\gamma; \obs)\propDens_{\Par}(\tilde{\Par} \given \Par)} \right\}
\end{equation}

This algorithm is recognized as a pseudo-marginal MCMC scheme targeting $\postApproxNormMarg$.
We could have alternatively arrived at this method by noting that $\postDens(\Par; \gamma)$ is an unbiased 
estimator of $\postEm(\Par)$ when $\gamma \sim \emDist(\cdot \given \Par)$, and then applying the standard
pseudo-marginal machinery. The pseudo-marginal method
can be viewed as a noisy alteration of the standard Metropolis-Hastings algorithm, such that exact evaluations 
of $\postDens(\Par)$ are replaced with realizations of $\postEm(\Par)$. The key to ensuring invariance with 
respect to $\postApproxNormMarg$ is that only a new sample of $\postEm(\tilde{\Par})$ is drawn at each iteration; 
the sampled value of $\postEm(\Par)$ is recycled from the previous iteration. The algorithm can be made more 
efficient (while still targeting $\postApproxNormMarg$), by averaging multiple sampled values. In particular, in place
of $\postDens(\propPar)$, we use 
$\hat{\postDens}^M(\propPar) \Def M^{-1} \sum_{m=1}^{M} \postDens(\propPar; \targetTraj^{(m)})$ where 
$\targetTraj^{(m)} \overset{\mathrm{iid}}{\sim} \emDist(\cdot; \tilde{\Par})$.

The pseudo-marginal approach to sampling from the EL approximation is noted in \citep{StuartTeck1} and studied 
in depth in \citep{garegnani2021NoisyMCMC}. The latter also studies a closely-related, but approximate, algorithm
whereby the estimators $\hat{\postDens}^M(\Par)$ and $\hat{\postDens}^M(\propPar)$ are both sampled independently
each iteration, rather than recycling the value of $\hat{\postDens}^M(\Par)$ from the previous iteration. This approach, 
typically known as \textit{Monte Carlo within Metropolis-Hastings (MCwMH)}, does not directly target $\postApproxNormMarg$
but is known to typically be more efficient \citep{noisyMCMC,stabilityNoisyMH,noisyMCSurvey}.
The limiting distribution of MCwMH (if it exists) depends on the value of $M$. As $M \to \infty$, the limiting distribution 
will tend towards $\postApproxNormMarg$. \citep{FerEmulation} utilize a MCwMH scheme with $M = 1$, adopting an
approximate computation viewpoint rather than explicitly trying to target $\postApproxNormMarg$.



\paragraph{TODO: some themes to stress throughout this section:}
\begin{enumerate}
\item Approximate computation approaches (no target density defined beforehand)
\item How computation differs between log-density and forward model emulator approaches.
\item Whether algorithms are applicable to the infinite-dimensional emulator case (e.g., GPs)
\end{enumerate}

\subsection{Expected Posterior}
\subsubsection{Direct Sampling}
Direct sampling, MCMC within Monte Carlo, Just-in-time sampling

\subsection{Approximate MCMC Methods}



\section{Temporary}
\subsection{Expected Likelihood}
The expected likelihood approximation may also be motivated via the definition of a modified 
inverse problem with an extended parameter space. To this end, we introduce a random variable
$\gamma$ such that $\targetEm[\Ndesign](\Par) \overset{\mathrm{d}}{=} \gamma \given \Par$.
Consider the joint probability model 
\begin{align}
p(\Par, \obs, \gamma) \Def 
p(\obs \given \gamma) p(\gamma \given \Par) \priorDens(\Par). \label{eq:EL-prob-model}
\end{align}
For a forward model emulator $p(\obs \given \gamma)$ is the likelihood evaluated at
the forward model output $\gamma$. For a log-likelihood emulator, 
$p(\obs \given \gamma) = \Exp{\gamma}$. The expected likelihood approximation 
corresponds to the conditional marginal 
\begin{align}
p(\Par \given \obs) \propto \priorDens(\Par) \int p(\obs \given \gamma) p(\gamma \given \Par) d\gamma
\label{eq:EL-marginal}
\end{align}
This extended parameter space formulation is adopted in \citet{BilionisBayesSurrogates}, noted in
\citet{SinsbeckNowak}, and a special case is given as an example in \citet{StuartTeck2}.

When the integral in \Cref{eq:EL-marginal} can be computed analytically, then $p(\Par \given \obs)$ can 
be targeted by any standard MCMC sampler. If samples can be drawn from $p(\gamma \given \Par)$ then
it can still be targeted exactly via pseudo-marginal methods. A third option is to update the marginal conditionals
\begin{align}
p(\Par \given \gamma, \obs) &\propto p(\gamma \given \Par) \priorDens(\Par) \label{eq:EL-conditionals} \\
p(\gamma \given \Par, \obs) &\propto p(\obs \given \gamma) p(\gamma \given \Par)
\end{align}
in turn, which can be done via Metropolis-Hastings steps, slice sampling, or other means. This algorithm
requires the ability to evaluate the predictive density of the emulator.

\subsection{Expected Posterior}
The probability model in \Cref{eq:EL-prob-model} does not encode the dependence structure of the 
surrogate predictive distribution across different values of $\Par$; the conditionals $\gamma \given \Par$
only capture the pointwise predictions. In this section, we consider formulating a probability model that 
reflects the full distribution of the random function $\targetEm$. Formally, consider
\begin{align}
p(\Par, \obs, \targetEm) &\Def
p(\Par \given \obs, \targetEm) p(\targetEm) p(\obs) \\
&= \priorDens(\Par) p(\obs \given \targetEm(\Par))p(\targetEm) \frac{p(\obs)}{p(\obs \given \targetEm)}, \label{eq:EP-prob-model}
\end{align}
where $p(\Par \given \obs, \targetEm)$ is the posterior over $\Par$ obtained using a particular 
trajectory of the emulator, with associated normalizing constant $p(\obs \given \targetEm)$.
The quantity $p(\targetEm)$ is purely formal at the moment, as no Lebesgue 
density exists for the random element $\targetEm$. This could be given meaning in \Cref{eq:EP-prob-model}
by replacing $\targetEm$ with a random vector representing a discretization of $\targetEm$. The expected 
posterior corresponds to the conditional marginal
\begin{align}
p(\Par \given \obs) \propto \int p(\Par \given \obs, \targetEm) p(\targetEm) d\targetEm.
\label{eq:EP-marginal}
\end{align}
A sample can be directly drawn from this distribution by first sampling a trajectory $\target \sim \mathrm{law}(\targetEm)$
and then drawing a sample from $p(\Par \given \obs, \target)$. Assuming the first step can be completed, 
then the second step will still typically require an MCMC run. This approach is thus parallelizable, but 
may require many MCMC runs to adequately characterize the expected posterior. The larger problem is 
that the first step is often intractable without resorting to discretization of $\targetEm$. An alternative to direct
sampling is to target the joint posterior over $(\Par, \targetEm)$ within a single MCMC run. We might try to 
do this analogously to \Cref{eq:EL-conditionals} by updating the marginal conditionals 
\begin{align}
p(\Par \given \targetEm, \obs) \propto \priorDens(\Par) p(\obs \given \targetEm(\Par)) \label{eq:EP-conditionals} \\
p(\targetEm \given \Par, \obs) \propto p(\obs \given \targetEm(\Par)) \frac{p(\targetEm)}{p(\obs \given \targetEm)}
\end{align}
in turn, which is most simply done via Metropolis-Hastings updates.
The $\Par$ update should only require a standard Metropolis-Hastings step. We focus on the
$\target$ update.

 Assuming a proposal 
$\tilde{\target} \sim q_{\target}(\target, \cdot)$, the ratio appearing in the MH acceptance
probability for the $\targetEm$ update takes the form 
\begin{align}
\frac{p(\obs \given \tilde{\target}(\Par))}{p(\obs \given \target(\Par))} \frac{p(\tilde{\target})}{p(\target)} 
\frac{q_{\target}(\tilde{\target}, \target)}{q_{\target}(\target, \tilde{\target})} \frac{p(\obs \given \target)}{p(\obs \given \tilde{\target})}.
\end{align}
There are two issues to deal with here. The first is the proposal and the infinite dimensional nature of $\targetEm$.
However, if we choose $q_\target$ to be $\mathrm{law}(\targetEm)$-invariant then the ratio should simplify to
\begin{align}
\frac{p(\obs \given \tilde{\target}(\Par))}{p(\obs \given \target(\Par))} \frac{p(\obs \given \target)}{p(\obs \given \tilde{\target})}.
\label{eq:MH-ratio-prior-invariant}
\end{align}
The simplest option is to set $q_\target(\target, \cdot) \Def \mathrm{law}(\targetEm)$, which yields an MH independence
sampler update. If the emulator $\targetEm$ depends on a transformation of a latent Gaussian, then 
specialized alternatives such as the preconditioned Crank-Nicolson proposal, or an elliptical slice
sampling update, may be considered. We emphasize that the simplification in 
\Cref{eq:MH-ratio-prior-invariant} means that this ratio only depends on the emulator trajectories at the current 
value of $\Par$. Thus, there is no requirement to sample full trajectories. See \Cref{alg:mwg-ep}.

The second issue is the dependence on the intractable normalizing constants 
$p(\obs \given \target)$ and $p(\obs \given \tilde{\target})$. 
If the proposed value $\tilde{\target}$ is not too far from $\target$ then it may be reasonable to invoke the 
approximation $p(\obs \given \target) / p(\obs \given \tilde{\target}) \approx 1$.
\Cref{alg:mwg-ep} uses this approximation, along with the independence MH proposal.

\begin{algorithm}
    \caption{Metropolis-within-Gibbs Approximation to $\llikEmSampDensNorm$}
    \label{alg:mwg-ep}
    \begin{algorithmic}[1] 
    \State \textbf{Input:} Current state $(\Par, \target_\Par)$
    \State \textbf{Output:} Updated state $(\Par^\prime, \target^\prime_{\Par^\prime})$
     \State Propose $\tilde{\Par} \sim q(\Par, \cdot)$
     \State Propose $(\tilde{\target}_\Par, \tilde{\target}_{\propPar}) \sim \mathrm{law}(\targetEm(\Par), \targetEm(\propPar))$ \Comment{Begin $\target$ update}
     \State $\alpha_{\target} \gets \min\left\{1, p(\obs \given \tilde{\target}_\Par) / p(\obs \given \target_\Par) \right\}$
      	\If{$\mathrm{Unif}(0,1) < \alpha_{\target}$}
                \State $(\target^\prime_\Par, \target^\prime_{\propPar}) \gets (\tilde{\target}_\Par, \tilde{\target}_{\propPar})$ 
            \Else
                \State $\target^\prime_{\Par} \gets \target_{\Par}$
                \State $\target^\prime_{\propPar} \gets \mathrm{law}(\targetEm(\propPar) \given \targetEm(\Par) = \target_\Par)$
            \EndIf
      \State $\alpha_{\Par} \gets \min\left\{1, \priorDens(\propPar )p(\obs \given \target^\prime_{\propPar}) / \priorDens(\Par) p(\obs \given \target^\prime_{\Par}) \right\}$ \Comment{Begin $\Par$ update}
             \If{$\mathrm{Unif}(0,1) < \alpha_{\Par}$}
                \State $\Par^\prime \gets \propPar$
                \State $\target^\prime_{\Par^\prime} \gets \target^\prime_{\propPar}$ 
            \Else
            	\State $\Par^\prime \gets \Par$
		\State $\target^\prime_{\Par^\prime} \gets \target^\prime_{\Par}$
            \EndIf
    \end{algorithmic}
\end{algorithm}

A second option is to consider an $\target$ update based on algorithms designed 
for doubly intractable target distributions. 














\subsection{Noisy Algorithm}

A common solution to this computational bottleneck is to learn a statistical approximation of the map 
$\Par \mapsto \postDens(\Par)$, which we refer to synonymously as a \textit{surrogate} or \textit{emulator}
model. Learning the surrogate typically consists of an initial offline step where training data 
is generated by running the expensive simulator at a set of input parameter values. A regression model 
is then fit to this training set to learn an approximation of the input-output relationship 
$\Par \mapsto \postDens(\Par)$. Our interest is in \textit{probabilistic} surrogate models, which output
a prediction for $\postDens(\Par)$ in the form of a probability distribution, as opposed to just a point estimate.
We let $\postEm[\Ndesign](\Par)$ denote the random variable representing the surrogate prediction 
of $\postDens(\Par)$, with the subscript indicating the number of exact simulation runs that were used 
to train the surrogate. We refer to the distribution of $\postEm[\Ndesign](\Par)$ as the surrogate 
\textit{predictive distribution} at input $\Par$; we allow this distribution to be arbitrary, and potentially 
only accessible through samples. Note that the map $\Par \mapsto \postDens(\Par)$ is typically not 
approximated directly; rather, an emulator is fit to a quantity on which $\postDens(\Par)$ depends 
(e.g., the log-likelihood or an underlying mechanistic model), which then induces a random approximation 
of $\postDens(\Par)$. Our framework thus seeks to be agnostic to the particular quantity that is 
being emulated. Concrete examples of typical surrogate modeling workflows are given in \cref{sec:surrogate-examples}.

Given $\postEm[\Ndesign]$, a fixed random approximation of $\postDens$, a natural question is how to 
utilize this surrogate in approximating the posterior distribution in \Cref{eq:post_dens_generic}. \Cref{sec:post-approx}
is dedicated to this question, in which we summarize previous approaches and present our proposed method.




We focus on the Bayesian inference setting in which pointwise evaluations of the unnormalized posterior

are available, but expensive owing to the cost of computing the likelihood $p(\obs \given \Par)$. This setting 
commonly arises in the Bayesian approach to inverse problems \citep{Stuart_BIP}. 
Consider a \textit{forward model} $\fwd: \parSpace \subseteq \R^{\dimPar} \to \R^{\dimObs}$ describing 
some system of interest, parameterized by input parameters $\Par \in \parSpace$. In addition, suppose that 
we have noisy observations $\obs \in \obsSpace \subseteq \R^{\dimObs}$ of the output signal that $\fwd(\Par)$ 
seeks to approximate. The \textit{inverse problem} concerns learning the parameter values $\Par$ such
 that $\fwd(\Par) \approx \obs$; i.e., \textit{calibrating} the model so that it agrees with the observations. 
 The statistical approach to this problem assumes that the link between model outputs and 
observations is governed by a probability distribution on $\obs \given \Par$. We assume that this distribution 
admits a density with corresponding log-likelihood 
\begin{align}
\llik: \parSpace \to \R, \label{log_likelihood}
\end{align}
such that $p(\obs \given \Par) = \Exp{\llik(\Par)}$. The notation $\llik(\Par)$ suppresses the dependence 
on $\obs$, as the observed data will be fixed throughout. We start by focusing inference only on the 
calibration parameter $\Par$, assuming that other likelihood parameters (e.g., noise covariance) 
are fixed. We discuss inference for such nuisance parameters in \Cref{section_lik_par}.

The Bayesian approach completes this specification with a prior distribution 
on $\Par$. Letting $\priorDens(\Par)$ denote the density of this distribution, the Bayesian solution of the inverse problem is given 
by the posterior distribution
\begin{align}
\postDensNorm(\Par) \Def p(\Par \given \obs) = \frac{1}{\normCst}\postDens(\Par) = \frac{1}{\normCst} \priorDens(\Par) \Exp{\llik(\Par)}, \label{post_dens}
\end{align}
with normalizing constant given by 
\begin{align}
\normCst &= \int \priorDens(\Par) \Exp{\llik(\Par)} d\Par. \label{norm_cst}
\end{align}
We emphasize that throughout this paper $\postDens(\Par)$ denotes the \textit{unnormalized} 
posterior density. We will also find it useful to introduce the notation
\begin{equation}
\lpost(\Par) \Def \log \postDens(\Par) = \log \priorDens(\Par) + \llik(\Par) \label{eq:lpost}
\end{equation}
for the logarithm of the unnormalized posterior density.
When relevant, we will make explicit the dependence on the forward 
model by writing $\llik(\Par; \fwd)$, $\lpost(\Par; \fwd)$, and $\postDens(\Par; \fwd)$. 

In addition to considering 
generic likelihoods, we will give special attention to the additive Gaussian noise model
\begin{align}
\obs &= \fwd(\Par) + \noise \label{inv_prob_Gaussian} \\
\noise &\sim \Gaussian(0, \likPar) \nonumber 
\end{align}
with corresponding log-likelihood 
\begin{align}
\llik(\Par; \fwd) &= \log\Gaussian(\obs| \fwd(\Par), \likPar) 
= -\frac{1}{2} \log\det (2\pi \likPar) - \frac{1}{2} (\obs - \fwd(\Par))^\top \likPar^{-1} (y - \fwd(\Par)), \label{llik_Gaussian}
\end{align}
as this model features prominently in the inverse problems literature.
In general, the nonlinearity of $\fwd$ precludes a closed-form characterization of the posterior. In this case, the 
standard approach is to instead draw samples from the distribution using a Markov chain Monte Carlo (MCMC) 
algorithm. Such algorithms are serial in nature, often requiring $\sim 10^5 - 10^7$ iterations, with each 
iteration involving an evaluation of the unnormalized posterior density $\postDens(\Par; \fwd)$. 
In the inverse problem context, this computational requirement is often prohibitive when the forward model 
evaluations $\fwd(\Par)$ incur significant computational cost, as each density evaluation requires the 
expensive computation $\fwd(\Par)$. Motivated by this problem, a large body of work has focused on deriving 
cheap approximations to either $\fwd(\Par)$ or $\lpost(\Par)$ (note that approximating the former induces 
an approximation of the latter). We focus on surrogate models that take the form of statistical 
regression models approximating the maps $\Par \mapsto \fwd(\Par)$ or $\Par \mapsto \lpost(\Par)$.
\footnote{This is in contrast to other surrogate modeling strategies (e.g., reduced-order modeling)
that exploit the specific structure of the forward model; see, e.g., \todo.}
We refer to methods that explicitly model the former map as \textit{forward model emulation}, and those that 
model the latter as \textit{log-density emulation}. We also include methods that emulate the log-likelihood
map $\Par \mapsto \llik(\Par)$ in this latter category (see \Cref{sec:llik_vs_lpost}).
Throughout this review, we will typically view the functions
$\fwd(\Par)$, $\llik(\Par)$, and $\lpost(\Par)$ as computationally expensive black-boxes. We will occasionally refer to 
such maps as \textit{simulators}, owing to the fact that in typical applications evaluating these maps requires
running an expensive computer code. The following section provides a concrete example of such a simulation
model stemming from the numerical solution of differential equations.


\bibliography{prob_surrogates_bayes} 
% \bibliographystyle{ieeetr}

\end{document}







