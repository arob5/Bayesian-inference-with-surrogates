%!TEX root = SMA-book.tex

% Useful info on newcomand: https://tex.stackexchange.com/questions/117358/newcommand-argument-confusion


% Editing utilities (comments, todos)
\usepackage{comment}
\usepackage{xcolor}
\newcommand\issue[1]{\textcolor{red}{#1}}
% \renewcommand\issue[1]{} % Uncomment to turn off red highlighting
\usepackage{todonotes}

% --------------------------------------------------------------------------------------------------
% General commands.  
% --------------------------------------------------------------------------------------------------

\newcommand{\todo}{\textbf{TODO }}


% --------------------------------------------------------------------------------------------------
% General Math.  
% --------------------------------------------------------------------------------------------------

% Common math commands. 
\newcommand*{\abs}[1]{\left\lvert#1\right\rvert}
\newcommand{\R}{\mathbb{R}}
\newcommand*{\suchthat}{\,\mathrel{\big|}\,}
\newcommand{\Exp}[1]{\exp\mathopen{}\left\{#1\right\}\mathclose{}}
\newcommand{\Log}[1]{\log\mathopen{}\left\{#1\right\}\mathclose{}}
\newcommand{\BigO}{\mathcal{O}}
\newcommand{\Def}{\coloneqq} % coloneqq comes from the mathtools package. 
\let\oldd\d\renewcommand\d{\relax\ifmmode\mathrm{d}\else\oldd\fi} %make \d be mathrm{d} in math mode, usual defn as underdot(?) in text mode
\newcommand{\proptoAdd}{\overset{\mathrm{add}}{\propto}} % Absorbs additive constants
\newcommand{\diag}{\textrm{diag}}
\newcommand{\indicator}[1]{\mathds{1}_{#1}}

\DeclareMathOperator*{\argmax}{argmax}
\DeclareMathOperator*{\argmin}{argmin}
\DeclarePairedDelimiterX\innerp[2]{(}{)}{#1\delimsize\vert\mathopen{}#2}
\DeclarePairedDelimiter{\ceil}{\lceil}{\rceil}

% Linear algebra.
\newcommand*{\norm}[1]{\left\lVert#1\right\rVert}
\newcommand{\bx}{\mathbf{x}}
\newcommand{\bw}{\mathbf{w}}
\newcommand{\oneVec}[1][]{\boldsymbol{1}_{#1}}
\newcommand{\idMat}{\mathbf{I}}
\newcommand{\bS}{\mathbf{S}}
\newcommand{\bK}{\mathbf{K}}
\newcommand{\ba}{\mathbf{a}}
\newcommand{\bb}{\mathbf{b}}

% Probability. 
\newcommand{\E}{\mathbb{E}}
\newcommand{\Var}{\mathrm{Var}}
\newcommand{\Cov}{\mathrm{Cov}}
\newcommand{\Cor}{\mathrm{Cor}}
\newcommand{\Prob}{\mathbb{P}}
\newcommand{\Gaussian}{\mathcal{N}}
\newcommand{\LN}{\mathrm{LN}} % Log-normal distribution 
\newcommand{\eqDist}{\overset{d}{=}} % Equality in distribution 
\newcommand{\given}{\mid} 
\DeclarePairedDelimiterX{\divergencex}[2]{(}{)}{%
  #1\;\delimsize\|\;#2%
}
\newcommand{\divergence}{\mathcal{D}\divergencex}
\newcommand{\KL}{\mathcal{D}_{\mathrm{KL}}\divergencex}
\newcommand{\law}{\textrm{law}}
\newcommand{\cv}{\mathsf{cv}} % coefficient of variation

% Other
\newcommand{\cst}{C}




